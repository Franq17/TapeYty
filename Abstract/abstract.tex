\abstract{
\addtocontents{toc}{\vspace{1em}} 
\textit{TapeYty} is a tool developed to calculate optimal paths for urban garbage collection vehicles in the Asunción city. This tool generates benefits mainly in the economic and environmental aspects of the city where a large number of people brings a high generation of waste. This makes the complexity of garbage management even greater. 

The routing problem is treated as the open rural postman problem which seeks to minimize the distance to be traveled by the collection vehicles in its working area. To achieve the objective \textit{TapeYty} is based on mathematical programming techniques and Geographical Informatiom System (GIS) tool which allows the management of the route network to be able to update the road way and their state of blocked or nonblocked. This implies that when changes of state of the streets \textit{TapeYty} modifies the graph that represents the road network and re-calculates the solutions providing new optimal routes to each vehicle of collection. 

The tool has provided solutions that save on average 20\% of distance traveled compared to current tour.
}