\documentclass[spanish, 12pt, oneside]{Thesis}  
% Use the "Thesis" style, based on the ECS Thesis style by Steve Gunn
\usepackage[spanish,activeacute]{babel}
\graphicspath{ThesisFigs/}  % Location of the graphics files (set up for graphics to be in PDF format)
\usepackage[utf8]{inputenc}
% \usepackage{algorithm2e}
\usepackage[Algoritmo]{algorithm}
\usepackage{algorithmic}

% Include any extra LaTeX packages required
%\usepackage[square, numbers, comma, sort&compress]{natbib}  % Use the "Natbib" style for the references in the Bibliography
\usepackage[natbibapa]{apacite}
\usepackage{verbatim}  % Needed for the "comment" environment to make LaTeX comments
\usepackage{vector}  % Allows "\bvec{}" and "\buvec{}" for "blackboard" style bold vectors in maths
\hypersetup{urlcolor=black, colorlinks=true}  % Colours hyperlinks in blue, but this can be distracting if there are many links.
\usepackage{mathtools}
\usepackage{amsmath}
\usepackage{mathrsfs}
\usepackage{fancyhdr}
\usepackage{nomencl}
\usepackage{float}
\usepackage[left]{lineno}
\usepackage[titletoc]{appendix}
\usepackage{graphicx}
\usepackage{caption}
\usepackage{subfigure}
\usepackage[export]{adjustbox}
\usepackage{wrapfig}
\usepackage{standalone}
\usepackage{tikz}
\usetikzlibrary{shapes,arrows,calc,positioning}
\usetikzlibrary{babel}
\usepackage{color}
\definecolor{mylightblue}{RGB}{059,131,189}
\definecolor{mygreen}{RGB}{040,114,051}
\definecolor{myblue}{RGB}{062,095,138}
\definecolor{mygray}{RGB}{156,156,156}
\definecolor{myorange}{RGB}{255,128,0} 

\tikzstyle{intt}=[draw,text centered,minimum size=6em,text width=5.25cm,text height=0.34cm]
\tikzstyle{intl}=[draw,text centered,minimum size=2em,text width=2.75cm,text height=0.34cm]
\tikzstyle{int}=[draw,minimum size=2.5em,text centered,text width=3.5cm]
\tikzstyle{intg}=[draw,minimum size=3em,text centered,text width=6.cm]
\tikzstyle{sum}=[draw,shape=circle,inner sep=2pt,text centered,node distance=3.5cm]
\tikzstyle{summ}=[drawshape=circle,inner sep=4pt,text centered,node distance=3.cm]
\tikzstyle{decision} = [draw,shape=diamond, fill=blue!20, text width=6em, text badly centered, node distance=3cm, inner sep=0pt]

\usepackage{varioref}
\usepackage[tablename=Tabla]{caption}

\setlength\unitlength{20pt}


\usepackage{pst-plot}
\addtopsstyle{gridstyle}{gridlabels=0pt,griddots=0}
%% ----------------------------------------------------------------
\newcommand{\abbrlabel}[1]{\makebox[3cm][l]{\textbf{#1}\ }}
\newenvironment{abbreviations}{
    \begin{list}{}
    {
    \renewcommand{\makelabel}{\abbrlabel}
    \setlength\leftmargin{1.5in}
    \setlength\labelwidth{1.5in}
    }
}{
    \end{list}
}
%% ----------------------------------------------------------------

% Command to use grados
\newcommand{\grad}{$^{\circ}$ }

% -----------------------------------------------------------
\begin{document}

\maketitle
%% ----------------------------------------------------------------

\setstretch{1.3}  % It is better to have smaller font and larger line spacing than the other way round
\frontmatter      % Begin Roman style (i, ii, iii, iv...) page numbering
% Define the page headers using the FancyHdr package and set up for one-sided printing
\fancyhead{}  % Clears all page headers and footers
\rhead{\thepage}  % Sets the right side header to show the page number
\lhead{}  % Clears the left side page header

\pagestyle{fancy}  % Finally, use the "fancy" page style to implement the FancyHdr headers


%% --------------- Dedicatorias --------------------------------

\setstretch{1.3}  % Return the line spacing back to 1.3

\pagestyle{empty}  % Page style needs to be empty for this page
%\dedicatory{For/Dedicated to/To my\ldots}

\dedicatory{
Dedicatoria
% \begin{flushright}
% Dedico este trabajo a mis padres Luis y Felipa, que con su amor y trabajo me educaron y apoyaron en toda mi formación profesional. \newline
% A mis hermanas, por su apoyo incondicional y sus palabras de aliento en este largo trayecto recorrido.

% \textbf{Andrea}
% \end{flushright}


% \begin{flushright}
% Este trabajo va dedicado a mis padres, Ricardo Mora y Nilda Britos, mis pilares fundamentales a lo largo de mi formación profesional y personal. Gracias por todo.\newlin% % A mis compañeros y grandes amigos con quienes he pasado una gran parte de mi vida en estos años de formación profesional, gracias por el tiempo dedicado en los momentos que necesité.
% \newline
% Gracias a Dios y María, por ser una hija bendecida y permitirme culminar esta etapa para comenzar una nueva.

% \textbf{Francisco}
% \end{flushright}
}

 
   % Add a gap in the Contents, for aesthetics


%% --------------- Agradecimientos --------------------------------
% The Acknowledgements page, for thanking everyone
 \acknowledgements{
\addtocontents{toc}{\vspace{1em}}  % Add a gap in the Contents, for aesthetics


A la profesora M.Sc. Ing María Elena García por su valiosa orientación, recopilación de información y sus palabras de aliento en todo el proceso de realización de este trabajo de fin de grado. 

Al profesor Lic. Jorge Meza por su orientación, por compartir sus conocimientos en el área GIS y sus distintos puntos de vista a lo largo de este trabajo.

Al D.Sc. Ing. Diego P. Pinto-Roa por sumarse a este trabajo y estar presente ante nuestras consultas e inquietudes siempre que se las presentábamos.

Al Ing. Pedro Bogado de la Dirección de Servicios Urbanos por brindarnos los datos y requerimientos necesarios para el desarrollo de este trabajo.

A los cuatro (4), un agradecimiento especial por el tiempo brindado para que pudiésemos culminar una etapa importante en nuestras vidas.

A todos los docentes que con su sabiduría y conocimiento, motivaron a desarrollarnos como personas y profesionales.

A nuestros padres por todo el amor, sacrificio y trabajo que nos dedicaron para que podamos lograr todos nuestros objetivos.

}
 \clearpage  % End of the Acknowledgements

%% --------------- Abstract ---------------------------------------

\addtotoc{Abstract}  % Add the "Abstract" page entry to the Contents
\abstract{
\addtocontents{toc}{\vspace{1em}}

\textit{TapeYty} is a tool developed to calculate optimal paths for urban garbage collection vehicles in the Asunción city. This tool generates benefits mainly in the economic and environmental aspects of the city where a large number of people brings a high generation of waste. This makes the complexity of garbage management even greater. 

The routing problem is treated as the directed open rural postman problem which seeks to minimize the distance to be traveled by the collection vehicles. To achieve the objective \textit{TapeYty} is based on mathematical programming techniques and Geographical Informatiom System (GIS) tool which allows the management of the route network, being able to update the road way and their state of blocked or nonblocked. This implies that when changes of street states are submitted  \textit{TapeYty} modifies the graph that represents the road network and re-calculates the solutions providing new optimal routes to each collection vehicle. 

The tool has provided solutions that save on average 20\% of distance compared to current tour.


% \textit{TapeYty} is a tool developed to calculate optimal paths for urban garbage collection vehicles in the Asunción city. This tool generates benefits mainly in the economic and environmental aspects of the city where a large number of people brings a high generation of waste. This makes the complexity of garbage management even greater. 

% The routing problem is treated as the open rural postman problem which seeks to minimize the distance to be traveled by the collection vehicles in its working area. To achieve the objective \textit{TapeYty} is based on mathematical programming techniques and Geographical Informatiom System (GIS) tool which allows the management of the route network to be able to update the road way and their state of blocked or nonblocked. This implies that when changes of state of the streets \textit{TapeYty} modifies the graph that represents the road network and re-calculates the solutions providing new optimal routes to each vehicle of collection. 

% The tool has provided solutions that save on average 20\% of distance traveled compared to current tour.
}
\clearpage  % Abstract ended, start a new page

%% --------------- Resumen --------------------------------------
\addtotoc{Resumen}  % Add the "Abstract" page entry to the Contents
\begin{resumen}
En la ciudad Asunción, la cantidad de personas que convergen a diario trae consigo una alta generación de residuos y esto hace que la complejidad en la gestión de la basura sea cada vez mayor. Se aborda el problema de enrutamiento como un problema del cartero rural abierto dirigido, minimizando la distancia a recorrer por los vehículos recolectores y luego se realiza una búsqueda en profundidad para obtener su secuencia. Para lograr el objetivo \textit{TapeYty} se basa en técnicas de programación matemáticas y herramienta GIS lo que permite la gestión de la red de rutas al poder actualizar el sentido de las calles, inhabilitar calles, agregar restricciones de giro o de continuar el sentido (contramano). Esto implica que ante cambios de estado de las calles \textit{TapeYty} re-calcula las soluciones y provee el nuevo recorrido óptimo a cada vehículo de recolección.
\end{resumen}
\clearpage  % Abstract ended, start a new page
%% ----------------------------------------------------------------

\pagestyle{fancy}  %The page style headers have been "empty" all this time, now use the "fancy" headers as defined before to bring them back


%% --------------- Indice -----------------------------------------
\lhead{\emph{Contenido}}  % Set the left side page header to "Contents"
\tableofcontents  % Write out the Table of Contents

%% --------------- Lista de Figuras -------------------------------
 \lhead{\emph{Lista de Figuras}}  % Set the left side page header to "List if Figures"
\renewcommand\listfigurename{Lista de Figuras}
 \listoffigures  % Write out the List of Figures
 
  
%% --------------- Lista de Acrónimos -----------------------------
\addtotoc{Lista de Acrónimos}
\lhead{\emph{Acrónimos}}  % Set the left side page header to "Symbols"
\chapter*{Lista de Acrónimos\hfill}
\begin{abbreviations}
    \item[DSU] \textbf{D}irección de \textbf{S}ervicios \textbf{U}rbanos.
    \item[SWM] \textbf{S}olid de \textbf{W}aste \textbf{M}anagement.
    \item[TWL] \textbf{T}hreshold \textbf{W}aste \textbf{L}evel.
\end{abbreviations}

%% --------------- Lista de Simbolos ------------------------------
\addtotoc{Lista de Símbolos}
\lhead{\emph{Símbolos}}  % Set the left side page header to "Symbols"
\chapter*{Lista de Símbolos\hfill}

\begin{abbreviations}

    % cap4
    \item[$x$] Vector de variables.
    \item[$f$] Función objetivo de un problema de optimización.
    \item[$c_{i}$] Vector de restricciones.
    \item[$I$] Conjunto de índices.
    \item[$E$] Conjunto de índices.
    \item[$i$] Índices de E e I.
    \item[$\mathbb{R}$] Conjunto de números reales.
    \item[$A$] Matriz.
    \item[$m$ x $n$] Dimensiones de la matriz A (m números de filas, n números de columnas.
    
    % cap5
    \item[$H$] Grafo mixto.
    \item[$G$] Grafo dirigido.
    \item[$V$] Conjunto de vértices del grafo G.
    \item[$A$] Conjunto de arcos de la grafo G.
    \item[$E$] Subconjunto de A. Representan segmentos de calles de dos vías que pueden ser recorridos en cualquier sentido .
    \item[$(i,j)$] Representa un arco de A (segmentos de calle).
    \item[$AM$] Subconjunto de A. Segmentos de calles que deben recorrerse únicamente en el sentido especificado.
    \item[$w$] Función de peso que asocia un peso a cada arco de A.
    \item[$I$] Subconjunto de V. Posibles puntos iniciales de una ruta.
    \item[$S$] Subconjunto de V.
    \item[$V \backslash S$] Subconjunto de V que no pertenecen a S.
    \item[$\delta^+ (S)$] Subconjunto de A que van desde vértices en S a vértices en $V \backslash S$.
    \item[$x_{i j}$] Número de veces que un arco de A es atravesado en una ruta.
    \item[$s_i$] Especifica si un vértice en I es el primer nodo en una ruta.
    \item[$t_j$] Especifica si un vértice en V es el último nodo en una ruta.
    \item[$M$] Modelo relajado utilizado en la solución propuesta.
    \item[$MG$] Multigrafo dirigido.
    \item[$O$] Notación O grande.
    \item[$N$] número de vértices en MG
    \item[$M$] número arcos en MG
    
    
    % % cap2
    % \item[$(i,j)$] Coordenadas espaciales de una imagen.
    % \item[$F(i,j)$] Imagen definida como función de dos dimensiones.
    % \item[$f(i,j)$] Imagen digital.
    % \item[$f'(i,j)$] Imagen modificada al aplicar una técnica de mejora de contraste.
    % \item[$M$ x $N$] Dimensiones de una matriz (M números de filas, N números de columnas.
    % \item[$(c,l)$] Índices de M y N.
    % \item[$p$] Píxel de una imagen.
    % \item[$\mathcal{H}$] Histograma de una imagen.
    % \item[$L$] Cantidad total de niveles de gris disponibles en una imagen.
    % \item[$L-1$] Nivel máximo de gris en una imagen.
    % \item[$k$] Nivel de gris o k-ésimo nivel de gris.
    % \item[$k'$] Nuevo nivel de gris mapeado por la ecualización del histograma.
    % \item[$n_k$] Número de ocurrencia de la intensidad k en la imagen.
    % \item[$Z$] Cantidad total de píxeles de una imagen.
    % \item[$\mathcal{R}$] Region contextual de una imagen.
    % \item[$(\mathcal{R}_i,\mathcal{R}_j)$] Tamaño de una Region contextual de una imagen.
    % \item[$\mathscr{C}$] Clip Limit usado por el Algoritmo $CLAHE$.
    
    % % cap3
    % \item[$P^*$] Conjunto Pareto Óptimo.
    % \item[$FP^*$] Frontera Pareto.
    % \item[$\Omega$] Región de soluciones factibles para el conjunto Pareto Óptimo.
    % \item[$\rho$] Una partícula del enjambre.
    % \item[$\chi _i$] Vector que almacena la posición actual de cada partícula.
    % \item[$\rho{Best_i}$] Vector que almacena la posición de la mejor solución encontrada por la partícula $i$ hasta el momento.
    % \item[$v_i$] Vector que almacena la velocidad actual de cada partícula.
    % \item[$t$] Número de iteraciones del algoritmo $PSO$.
    % \item[$v^{t}_i$] Velocidad de la partícula $i$ en la iteración $t$.
    % \item[$\omega$] Coeficiente de inercia del $PSO$.
    % \item[$\varphi_1$] Máximo valor que puede alcanzar el coeficiente cognitivo del $PSO$.
    % \item[$\varphi_2$] Máximo valor que puede alcanzar el coeficiente social del $PSO$.
    % \item[$\chi^{t}_i$] Posición actual de la partícula $i$ en la iteración $t$.
    % \item[$g_i$] Posición de la partícula con el mejor $fitness_{\rho{Best_i}}$ del entorno de $\rho_i$ o de todo el enjambre.
    % \item[$\kappa$] Coeficiente de restricción aplicado a $v^{t}_i$.
    
    
    % %cap4
    % \item[$\mathcal{P}_k$] Probabilidad de ocurrencias del nivel de gris $k$ en el histograma utilizado para el cálculo de $\mathscr{H}$.
    % \item[$\nu_{ij}$]   Representa una vecindad de una imagen.
    % \item[$\mathscr{H}$] Entropía.
    % \item[$\mathscr E$] Entropía Local.
    
    %  % cap5
    %  \item[$\gamma$] Coeficiente de correlación.
    %  \item[$I_f$ y $S_f$] Funciones objetivos de la propuesta.
     
    % \item[$\overrightarrow{x}$] Vector que almacena los valores de [$\mathcal{R}_i$, $\mathcal{R}_j$, $\mathscr{C}$].
    % \item[$\Re$] Frente Pareto Robusto.
    % \item [$\mathscr X$] Conjunto de soluciones no dominadas.
    % \item[$\Gamma$] Frente Pareto.
    % \item[$I$] Conjunto de imágenes.
    % \item[$\alpha$] Conjunto de soluciones óptimas de $I$ analizadas.
    % \item[$\beta$] Promedio de las soluciones óptimas originales de las imagenes.
    % \item[$\delta$] Promedio de soluciones óptimas entre todas las imagenes analizadas.
    
\end{abbreviations}

%% --------------- Lista de Tablas --------------------------------
 \lhead{\emph{Lista de Tablas}}  % Set the left side page header to "List of Tables"
 \renewcommand\listtablename{Lista de Tablas}
 \listoftables  % Write out the List of Tables
 
%% --------------- Contenido -----------------------------------
\clearpage
\mainmatter	  % Begin normal, numeric (1,2,3...) page numbering
\pagestyle{fancy}  % Return the page headers back to the "fancy" style
% Include the chapters of the thesis, as separate files
% Just uncomment the lines as you write the chapters

\lhead{\emph{Introducción}} 
\renewcommand\chaptername{Capítulo}%título "Capítulo"
\chapter{Introducción}
\label{chap1}
\ifpdf
  \graphicspath{{Chapter1/Chapter1Figs/PNG/}{Chapter1/Chapter1Figs/PDF/}{Chapter1/Chapter1Figs/}}
\else
  \graphicspath{{Chapter1/Chapter1Figs/EPS/}{Chapter1/Chapter1Figs/}}
\fi

\markboth{\hfill \thechapter. Introducción}{\hfill \thechapter. Introducción}

El impacto ambiental que provocan los residuos sólidos municipales ha sido objeto de atención especial en las últimas décadas. La eliminación de los residuos sólidos urbanos es una preocupación creciente en todo el mundo, sin importar el tamaño ni las características socio-económicas de una ciudad. Muchas ciudades se han visto obligadas a evaluar su programa de gestión de residuos sólidos y examinar su relación costo-efectividad en términos de recolección, transporte, tratamiento y eliminación \citep{Karadimas2007OptimalAlgorithm}.

En la Gestión de Residuos Sólidos (SWM, \textit{Solid Waste Management}) se estima que de la cantidad total de dinero destinado para su recogida, transporte y eliminación, aproximadamente el 60-80\% se gasta en la fase de Recolección de Residuos Sólidos (SWC, \textit{Solid Waste Collection}) \citep{Tavares2009OptimisationModelling,Dogan2003Report:Istanbul}. Por lo general, la fase de recolección en los países en desarrollo se basa en la experiencia práctica y en métodos intuitivos, dando lugar a prácticas ineficientes y costosas, que afectan tanto a la salud pública como al medio ambiente. Por ende, incluso una pequeña mejora en la operación de recogida puede dar lugar a un ahorro importante en el costo total, motivo por el cual muchos municipios se han esforzado en mejorar la gestión de la basura.

La Municipalidad de Asunción, de ahora en adelante MDA, cuenta con un conjunto de programas de trabajo anuales. En la Figura \ref{fig:porcentajePresupuesto} se puede observar que la mayor parte del presupuesto total de la municipalidad correspondiente al año 2017 fue destinado a los programas de acción. A su vez, dentro de estos programas de acción fue el servicio de calidad en recolección y limpieza el que se llevó la mayor parte y con una diferencia significativa sobre las demás, como se muestra en la Figura \ref{fig:programaAccion}.

\begin{figure}[H]
    \centering
    \includegraphics[width=14.5cm]{20181119_PresupuestoTotal2017.png}
    \caption{Representación porcentual del Presupuesto Total de la Municipalidad de Asunción discriminado por los programas para el año 2017. [Fuente: Portal de la Municipalidad de Asunción - Ejercicio Fiscal 2017]}
    \label{fig:porcentajePresupuesto}
\end{figure}

\begin{figure}[H]
    \centering
    \includegraphics[width=15cm]{20181119_PresupuestoAccion2017.png}
    \caption{ Representación porcentual del Presupuesto correspondiente a los subprogramas del Programa de Acción correspondientes al año 2017. [Fuente: Portal de la Municipalidad de Asunción - Ejercicio Fiscal 2017]}
    \label{fig:programaAccion}
\end{figure}

\section{Antecedentes}

Cuando se habla de la problemática de la gestión de residuos sólidos, es sabido que existen numerosos estudios y trabajos científicos que se han realizado al respecto con el propósito de resolver usando objetivos económicos y ambientales como criterios para la toma de decisiones. Hasta la fecha, se siguen estudiando distintas técnicas que puedan permitir mejorar el proceso y como mencionan \citet{Tchobanoglous1993IntegratedIssues}, no existe un conjunto universal de reglas que puedan aplicarse a todas las situaciones.

% En la literatura, las soluciones al ruteo de la recolección de residuos sólidos urbanos se han planteado como distintos tipos de problemas, y están principalmente definidos en dos categorías: a) Problema de Enrutamiento de Vehículo Capacitado (CVRP, \textit{Capacitated Vehicle Routing Problem}), en donde se define una serie de nodos con demanda positiva cuyo objetivo es encontrar el mejor recorrido que cubra la totalidad de los nodos y; por otro lado, b) Problema de Enrutamiento de Arco Capacitado (CARP, \textit{Capacitated Arc Routing Problem}), en donde existe una serie de arcos predefinidos con demanda positiva o nula y el objetivo consiste en encontrar los mejores recorridos cubriendo todos los arcos requeridos \citep{Tirkolaee2018ATime}.

En la literatura, las soluciones al ruteo de la recolección de residuos sólidos urbanos se han planteado como distintos tipos de problemas, y están principalmente definidos en dos categorías \citep{Tirkolaee2018ATime}:

\begin{enumerate}[label=\alph*)]
    \item Problema de Enrutamiento de Vehículo Capacitado (CVRP, \textit{Capacitated Vehicle Routing Problem}), en donde se define una serie de nodos con demanda positiva cuyo objetivo es encontrar el mejor recorrido que cubra la totalidad de los nodos.
    \item Problema de Enrutamiento de Arco Capacitado (CARP, \textit{Capacitated Arc Routing Problem}), en donde existe una serie de arcos predefinidos con demanda positiva o nula y el objetivo consiste en encontrar los mejores recorridos cubriendo todos los arcos requeridos.
\end{enumerate}

% VRP CVRP VRPTW
En los enfoques que abordan el problema como un CVRP podemos citar a varios trabajos: \citet{Akhtar2017BacktrackingOptimization,Ombuki-Berman2007WASTEALGORITHMS,Kim2006WasteWindows,Billa2014GISOptimization,Karadimas2007OptimalAlgorithm}. En \citet{Akhtar2017BacktrackingOptimization} se desarrolló un Algoritmo de Búsqueda Hacia Atrás meta-heurístico (BSA, \textit{Backtracking Search Algorithm}) en un modelo CVRP con el concepto de contenedores inteligentes para encontrar las mejores soluciones de rutas de recolección. El estudio introduce el concepto de Umbral de Nivel de Residuos en contenedores (TWL, \textit{Threshold Waste Level}) para reducir el número de contenedores que deben ser vaciados al encontrar un rango óptimo de llenado, minimizando así la distancia, y consecuentemente reducir el combustible utilizado y las emisiones de gases.

\citet{Ombuki-Berman2007WASTEALGORITHMS} introduce el enfoque de un Algoritmo Genético multi-objetivo (GA, \textit{Genetic Algorithm}) para el enrutamiento de recolección de basura con ventanas de tiempo, minimizando el número total de vehículos y la distancia recorrida, atendiendo restricciones como capacidad de vehículo, ventanas de tiempo de parada y tiempos de almuerzo de conductores. El mayor potencial del algoritmo genético es que se puede utilizar en problemas prácticos de gran escala, buscando soluciones aproximadas en tiempo polinómico, en lugar de soluciones exactas que resultarían costosas para los de gran escala. Adoptó la definición del problema dado por \citet{Kim2006WasteWindows}, en donde se utilizó un algoritmo VRPTW de recolección de desechos basado en agrupación de nodos.

% TSP
Otros trabajos de la literatura tratan como un Problema del Vendedor Viajante (TSP, \textit{Travelling Sales Problem}), de hecho, el VRP surgió como una extensión del TSP para el caso en el que la capacidad de los vehículos que realizan la ruta sea limitada, siendo necesario realizar varios viajes \citep{CalvinoM2011CooperacionPanoramica}. En \citet{Billa2014GISOptimization} el enrutamiento óptimo se desarrolla con un método basado en el TSP y luego se integra con \textit{ArcInfo GIS} utilizando Programación Lineal (LP, \textit{Linear Programming}). Se reveló que las rutas óptimas pueden no ser necesariamente la distancia más corta desde el punto A al punto B, considerando la congestión del tráfico y la presencia de muchos semáforos en distancias cortas.

En \citet{Karadimas2007OptimalAlgorithm} se implementó un Sistema de Colonia de Hormigas (ACS, \textit{Ant Colony System}) para la identificación de rutas óptimas, que fue modelado como un TSP Asimétrico (ATSP, \textit{Asymmetric TSP}) para monitorear, simular, probar, y optimizar costos para diferentes escenarios de la WSM, donde un Sistema de Información Geográfica (GIS, \textit{Geographical Information System}) soporta la WSM usando parámetros como: la ubicación de cestos de basura, topología de red de carreteras, el tráfico relacionado y la densidad poblacional; además se consideran los horarios de recolección y las capacidades de los camiones.

En cuanto a la segunda categoría anteriormente mencionada, CARP, ha sido abordado por varios trabajos: \citet{Vecchi2016ACollection,Tirkolaee2018ATime,Braier2017AnArgentina}. En \citet{Vecchi2016ACollection} se presenta un enfoque secuencial para resolver el problema de optimización de la ruta de camiones recolectores. Se desarrolló un modelo para la solución del CARP, formulado como un Problema de Programación Lineal de Enteros Mixtos (MILP, \textit{Mixed Integer Linear Programming}), y luego se aplicó un algoritmo adaptado de Hierholzer para obtener la secuencia de los arcos. En \citet{Tirkolaee2018ATime} se presentó un interesane MILP para el problema de Enrutamiento de Arco Capacitado Periódico (PCARP, \textit{Periodic Capacitated Arc Routing Problem}) que tiene en cuenta los múltiples viajes, el tiempo de trabajo de los conductores y la tripulación para estudiar los efectos de la demanda incierta, para problemas de gran tamaño se aplicó un algoritmo híbrido basado en un algoritmo heurístico constructivo y un algoritmo de Recocido Simulado (SA, \textit{Simulated Annealing}).

% CPP, RPP, DRPP rural abierto.
Otro de los grandes problemas de arcos es el conocido como Problema del Cartero Rural (RPP, \textit{Rural Postman Problem}), que consiste en determinar el camino de mínima distancia que recorre solo algunos de los arcos del grafo y representa un problema \textit{NP-hard}, a no ser que el subgrafo formado por los arcos requeridos sea un grafo completamente conexo, en cuyo caso el RPP se reduce al Problema del Cartero Chino (CPP, \textit{Chinese Postman Problem}), para el cual se han definido algoritmos que lo resuelven en un tiempo polinómico \citep{CalvinoM2011CooperacionPanoramica}.

En \citet{Braier2017AnArgentina} se plantea un caso particular del RPP abierto Generalizado Dirigido (GDRPP, \textit{Generalized Directed Rural Postman Problem}). Desarrollaron un modelo de Programación de Enteros (IP, \textit{Integer Programming}) con un procedimiento de resolución basado en un algoritmo de mezcla de subtours y la adición dinámica de restricciones de eliminación de subtours. Este modelo sirvió de base para la implementación presentada en la siguiente sección.

% La herramienta ArcGIS Network Analytics (NA) es utilizada ampliamente en la búsqueda por minimizar la distancia y el tiempo de las rutas actuales de los camiones recolectores de residuos sólidos \citep{Kallel2016UsingTunisia, Malakahmad2014SolidMalaysia}. Existen además implementaciones comerciales similares a la solución que se propone, entre ellas se encuentran \textit{MapInfo Software}, \textit{RouteSmart}, \textit{Waste Route Software}, \textit{RouteViewPro}.
% % Agregar las implementacion libres.

% En este trabajo se propone el desarrollo de una herramienta, que optimice el camino seguido por los vehículos de recolección de basura de la ciudad de Asunción mediante técnicas de programación matemáticas, y consecuentemente, genere beneficios principalmente en los aspectos económicos y ambientales. Esta herramienta denominada de ahora en más \textit{TapeYty}, proveniente de dos palabras del idioma guaraní, \textit{Tape} que significa camino y \textit{Yty} que significa basura.

Varios modelos para la recolección y trasporte de los residuos sólidos han sido desarrollados basados en sistemas informáticos (\textit{software}) apropiados para la optimización de rutas. La herramienta ArcGIS Network Analytics (NA) es utilizada ampliamente en la búsqueda por minimizar la distancia y el tiempo de las rutas actuales de los camiones recolectores \citep{Kallel2016UsingTunisia,Malakahmad2014SolidMalaysia}. Existen además otras implementaciones comerciales, entre las que se encuentran \textit{MapInfo}, \textit{RouteSmart}, \textit{WasteRoute}, \textit{TransCAD}, \textit{RouteViewPro} mencionados en \citet{Kallel2016UsingTunisia} y \textit{Graphhopper} utilizado en \citet{Lozano2018SmartOptimization}.

Si bien los sistemas mencionados cumplen con el objetivo de optimizar los recorridos de vehículos, en su mayoría son de código cerrado y no se ajustan a los procedimientos específicos de una institución, resultando mucho más complejo y costoso extender sus funcionalidades. En este trabajo se propone el desarrollo de una herramienta sobre una arquitectura modular e interoperable, que optimice el camino a seguir por los vehículos de recolección de basura domiciliaria de la ciudad de Asunción mediante técnicas de programación matemáticas, y consecuentemente, genere beneficios principalmente en los aspectos económicos con la disminución de la distancia recorrida y en consecuencia el consumo de combustible; y ambientales con la disminución de la emisión de gases que dejan a su paso los camiones debido al menor tiempo de actividad \citep{Vu2018ParameterModel}. La herramienta denominada de ahora en adelante \textit{TapeYty}, proviene de dos palabras del idioma guaraní, \textit{Tape} que significa camino y \textit{Yty} que significa basura.

\section{Justificación}
 En la actualidad, los choferes de los vehículos recolectores de basura de la ciudad de Asunción trazan los caminos a seguir en base a su experiencia, razón por la cual es necesario optimizar el recorrido realizado para la recolección de residuos, reduciendo el costo y el tiempo de cada recorrido. Se propone desarrollar una aplicación que permita a la MDA gestionar de manera eficiente esos recorridos al poder actualizar el sentido de las calles, inhabilitar calles, agregar restricciones de giro o de contramano. La MDA podrá contar con una aplicación que garantice que todos los ciudadanos reciban el servicio de recolección domiciliaria.

El ingreso promedio de personas en ómnibus del transporte público y vehículos privados de los municipios aledaños, es alrededor de 1.320.000 a diario \citep{DiarioABCColor2016PorColor}, es así que no sólo los ciudadanos que residen en la ciudad de Asunción serán beneficiados con un servicio más eficiente, sino las miles de personas que ingresan diariamente a la ciudad, ya que al reducir el tiempo en tránsito del vehículo recolector se reducen los problemas de contaminación por los líquidos y gases que dejan a su paso.

Este proceso permitirá al personal encargado de la recolección de residuos agilizar su trabajo ya que pasarán menos tiempo en el vehículo recolector, mejorando así su calidad de vida.
 
Por todo lo expuesto, se considera plenamente justificada la investigación y propuesta de solución de este trabajo de fin de carrera, pues nos permitirá como ciudadanos retribuir en parte al Estado los beneficios y conocimientos que hemos adquirido a través de nuestra carrera en la Universidad.

\section{Objetivo General}
El objetivo general de este trabajo de investigación es el de proponer una solución que optimice el recorrido de los vehículos de recolección de basura domiciliaria de la Dirección Servicios Urbanos (DSU) de la MDA.

\section{Objetivos Específicos}

Los objetivos específicos que se han trazado en este trabajo son los siguientes:

\begin{enumerate}
    \item \textbf{Analizar} sobre técnicas de optimización para obtener el camino óptimo. 
    \item \textbf{Identificar} los factores que influyen en la recolección domiciliaria de la DSU.
    \item \textbf{Aplicar un modelo matemático} de optimización que mejor se ajuste a las reglas de negocio del caso de estudio.
    \item \textbf{Proponer y desarrollar una aplicación GIS} que permita configurar los parámetros de entrada del problema y despliegue la ruta óptima para cada zona de recolección.
    \item \textbf{Comparar} resultados de la aplicación desarrollada con los recorridos que actualmente son realizados, y de esta manera validar los resultados obtenidos.
\end{enumerate}


\section{Organización del Trabajo}

La organización del presente trabajo se completa de la siguiente manera:

\begin{itemize}
    \item En el \textbf{capítulo 2} se presentan los conceptos básicos que describen la gestión de residuos sólidos municipales y el caso de estudio, la ciudad de Asunción.
    \item En el \textbf{capítulo 3} se presentan conceptos básicos sobre Sistemas de Información Geográfica y bases de datos espaciales.
    \item En el \textbf{capítulo 4} se presentan las distintas técnicas de optimización.    
    \item En el \textbf{capítulo 5} se plantea de manera formal el problema que se intenta resolver. Se explican las formulaciones utilizadas, así como una explicación detallada de la implementación de la herramienta \textit{TapeYty}. 
    \item En el \textbf{capítulo 6} se visualizan y se discuten los resultados de la propuesta.
    \item En el \textbf{capítulo 7}, se finaliza el trabajo de investigación presentando las conclusiones generales, así como los aportes, aprendizajes y trabajos futuros continuando esta línea de investigación.
    % \item En el \textbf{capítulo ANEXOS}
\end{itemize} % Introduction

\clearpage
\lhead{\emph{Gestión de residuos sólidos}} 
\renewcommand\chaptername{Capítulo}%título "Capítulo"
\chapter{Gestión de residuos sólidos}
\label{chap2}
\ifpdf
  \graphicspath{{Chapter2/Chapter2Figs/}}
\else
  \graphicspath{{Chapter2/Chapter2Figs/}}\fi

\markboth{\hfill \thechapter. Gestión de residuos sólidos}{\hfill \thechapter. Gestión de residuos sólidos}

Se conoce como residuo a cualquier material en estado sólido, líquido o gaseoso resultante de los procesos de producción, transformación y utilización, que carente de valor para su propietario, éste decide abandonarlo.
%Agregar estadistica de genearacion diaria de residuos por personas de la ciudad de Asuncion

El decreto N$^{\circ}$ 7391, del 28 de junio de 2017, que reglamenta la Ley N$^{\circ}$ 3956/2009, acerca de la ``Gestión Integral de los Residuos Sólidos en la República del Paraguay", clasifica los residuos en:

\begin{itemize}
\item \textbf{Residuos sólidos urbanos:} Los generados en cada habitación, unidad habitacional o similares que resultan de la eliminación de los materiales que se utilizan en las actividades domésticas, de los productos que se consumen y de sus envases, embalajes o empaques, y los provenientes de cualquier otra actividad que genere residuos sólidos con características domiciliarias y los resultantes de la limpieza de las vías públicas y áreas comunes.
\item \textbf{Residuos de manejo especial:} Los generados en los procesos productivos que no reúnen las características para ser considerados como peligrosos o como residuos sólidos urbanos, o que son producidos por grandes generadores de residuos sólidos urbanos. Entre ellos se encuentran los provenientes de servicios de la salud, aquellos generados en los procesos productivos e instalaciones industriales y comerciales, también los generados por la actividades agrícolas, pesqueras y forestales, los de servicios de transportes, resultantes de las actividades que se realizan en terminales de transportes; también se incluyen los residuos de la construcción civil, los tecnológicos, los provenientes del tratamiento de aguas residuales, los neumáticos usados, muebles, los residuos de minería e hidrocarburos.
\item \textbf{Residuos peligrosos:} Resultantes de los procesos industriales y productos que han sido adquiridos y/o desechados, y que por sus características explosivas, inflamables, oxidantes, tóxicas, infecciosas, radioactivas, corrosivas, etc, pueden causar riesgos presentes o futuros a la calidad de vida de las personas o afectar el suelo, la flora, la fauna, contaminar el aire o las aguas de manera tal que dañen la salud humana o ambiental del país.
\end{itemize}

\section{Gestión de residuos sólidos}

La gestión de residuos sólidos (SWM, \textit{Solid Waste Management}) implica varios procesos que pueden dar lugar a problemas relacionados con áreas como la organización, el control, la logística, la planificación y el reciclaje. Es una actividad multidisciplinaria que contiene decisiones de criterios múltiples en cada etapa de su ciclo de vida. 

La gestión integral de residuos sólidos es el conjunto de acciones que se aplican en el manejo de los desechos desde su generación hasta su disposición final, basándose en criterios sanitarios, ambientales y de viabilidad técnica y económica para la reducción en la fuente de aprovechamiento, tratamiento y disposición final.

% Existen investigaciones en varios aspectos, como el área de pronóstico de generación de residuos, el monitoreo de los sistemas de recolección de contenedores, la gestión del transporte de contenedores y la predicción de la instalación de nuevas plantas de eliminación de residuos.

En el trabajo de \citet{VitorinodeSouzaMelare2017TechnologiesReview} se agrupan los procesos de SWM en seis categorías 
\begin{itemize}
\item Gestión de recogida, recorrido y transporte; 
\item Gestión y seguimiento de contenedores; 
\item Reciclaje de residuos sólidos y gestión de residuos electrónicos;
\item Administración pública y desarrollo sostenible; 
\item Métodos de previsión y planificación; y 
\item Determinación de sitios de disposición de residuos.
\end{itemize}

\subsection{La gestión de recogida, recorrido y transporte}
La recogida de los residuos consiste en su recolección para trasladarlos a su disposición final. Se distinguen dos tipos de recogida: selectiva y no selectiva.

La recogida selectiva consiste en agrupar y clasificar los residuos según sus características y propiedades con el fin de facilitar su tratamiento, en este tipo de recogida la ciudadanía tiene un rol esencial. En la recogida no selectiva los residuos se depositan en los contenedores y/o cestos de basuras sin ningún tipo de separación.

El transporte es la acción de trasladar los residuos sólidos de una fase de su gestión a otra, mientras que el recorrido se refiere al trayecto o camino que sigue el vehículo desde el inicio hasta el fin de sus actividades. Hay que tener en cuenta el problema que se asocia con el movimiento diario de los vehículos, este incide directamente sobre las calles, que deben estar adecuadamente acondicionadas, también en muchos casos es fuente de molestias para los vecinos debido a los malos olores, ruidos, tráfico, contaminación, entre otros.

\subsection{Gestión y seguimiento de contenedores}

Se pueden detectar dos escenarios: (1) áreas críticas con contenedores que siempre están llenos, que deben ser reemplazados con contenedores de mayor capacidad; o (2) áreas cuyos contenedores están siempre vacíos. Los costos destinados en el mantenimiento de los vehículos de recolección y combustibles pueden ser reducidos mediante tecnologías que controlen el nivel del contenedor, desplazándose así el vehículo solo cuando los contenedores se encuentren en el nivel adecuado para la recolección \citep{VitorinodeSouzaMelare2017TechnologiesReview}.

En la Figura \ref{fig:contenedoresMDA} se observan los contenedores móviles que se encuentran distribuidos en el microcentro de la ciudad de Asunción y en la zona de su costanera.

\begin{figure}[H]
    \centering
    \includegraphics[width=7cm]{contenedores_mda.png}
    \caption{Contenedores de la ciudad de Asunción. [Fuente: Diario Última Hora]}
    \label{fig:contenedoresMDA}
\end{figure}

\subsection{Reciclaje de residuos sólidos}

El reciclaje es el resultado de una serie de actividades a través de las cuales, materiales que se tornarían residuos, son desviados, siendo recolectados, separados y procesados para ser usados como materia prima en la producción de un nuevo producto de composición semejante.

Entre los beneficios y ventajas del reciclaje se encuentran: la preservación de los recursos naturales, menor contaminación, ahorro de energía, dinero y petróleo, además, fomenta el consumo responsable y genera empleos.

\subsection{Administración pública y desarrollo sostenible}

La gestión de los residuos sólidos en el Paraguay, así como en la mayoría de los demás países, recae en el fuero municipal, se debe contar con políticas y estrategias nacionales para el desarrollo sostenible de la misma. La ausencia de planificación e infraestructura inadecuada para la eliminación de residuos sólidos conduce a una gran cantidad de residuos que se descargan en áreas públicas sin preparación del suelo, como vertederos a cielo abierto y ríos.

\subsection{Métodos de previsión y planificación}

El proceso de desarrollo urbano conlleva el crecimiento poblacional, cambios en patrones de consumo e incremento en el ingreso, siendo éstos, los principales factores que explican el aumento en la generación de residuos sólidos domiciliarios \citep{Vasquez2005ModeloChile}, es por ello que es importante la elaboración de planes relacionados a la reducción de residuos sólidos, y la implementación de técnicas de previsión y modelos para predecir la generación de residuos, aumentando de esta manera la vida útil de los rellenos sanitarios en cuanto a su capacidad operativa o planes de expansión de instalaciones (tratamiento, traslado, y disposición).

\subsection{Determinación de sitios de disposición de residuos}

La selección de sitios para la disposición de residuos es un proceso complejo y considera criterios relacionados con el medio ambiente (suelo, características, declinación de la tierra, agua subterránea, ecosistema y geología), económico (presencia de caminos, distancia de las áreas residenciales, acceso al sitio y distancia de los centros de generación de residuos), y perspectivas sociales (aceptación de la población, proximidad a aeropuertos y sitios arqueológicos) \citep{Gbanie2013ModellingLeone}. Los métodos indiscriminados de eliminación de residuos han provocado la contaminación de cuerpos de agua, suelo y aire, lo que presenta importantes riesgos para la salud pública.

\section{Gestión de residuos sólidos urbanos en la ciudad de Asunción}

Asunción es la capital y la ciudad más poblada de la República del Paraguay. Es un municipio autónomo administrado como Distrito capital, cuenta con una superficie de 117 km$^{2}$, dividido actualmente en 68 barrios. Según las proyecciones de la \citet*{DireccionGeneraldeEstadistica2015Paraguay2000-2025} para el año 2019 se estima una población aproximada de 522.287 habitantes. Se debe agregar que el ingreso promedio de personas en ómnibus del transporte público y vehículos privados de los municipios aledaños, es alrededor de 1.320.000 a diario. Este análisis se hace en base a cálculos estimativos de la Unidad Coordinadora del Programa Metrobús del Ministerio de Obras Públicas y Comunicaciones (MOPC) \citep{DiarioABCColor2016PorColor}. La cantidad de personas que convergen diariamente en Asunción trae consigo una alta generación de residuos, esto hace que la complejidad en la gestión de los mismos sea cada vez mayor.

La DSU de la MDA es la encargada de la regulación y prestación de servicios de aseo, de recolección, disposición y tratamiento de los residuos del municipio, así como también del equipamiento, mantenimiento, limpieza y ornato de la infraestructura pública del municipio, incluyendo las calles, avenidas, parques, plazas, balnearios y demás lugares públicos. Según Rodrigo Velázquez, director de la DSU, el promedio diario normal de basura recogida suele ser entre 800.000 y 900.000 kilos \citep{LaNacion2016AsuncionBasura} [Dato obtenido en 9 de Enero de 2016]. 
% El Dato obtenido tiene la fecha de la publicacion en el diario, no la fecha de acceso
%Se tiene la cantidad exacta de zonas ahora
Actualmente el Departamento de Recolección de la DSU es responsable de la recolección de residuos sólidos urbanos, para ello divide la ciudad en zonas, según datos brindados por la DSU la ciudad está dividida en 134 zonas. El municipio cuenta con contenedores distribuidos por el microcentro capitalino, estos son recogidos de la misma manera que los residuos domiciliarios. Mínimamente debe existir una calle empedrada para que los vehículos recolectores recojan los residuos, es por esto que lugares como los ``Bañados Norte y Sur" no son beneficiados con el servicio.

\subsection{Procedimiento de recolección de residuos urbanos en la Municipalidad de Asunción}

A continuación se detalla el procedimiento llevado a cabo habitualmente para la recolección de los residuos domiciliarios:

\begin{itemize}
\item Se establecen los días y frecuencias de recolección para cada zona.
\item Un equipo de trabajo cuenta con un chofer y tres recolectores, que es asignado a un vehículo y una zona. El vehículo es identificado por su chapa o por el número de identificación del camión proveído por la DSU. Estos vehículos recolectores son de propiedad de la MDA, en caso de que uno de ellos quede fuera de servicio por algún desperfecto, por lo general, la solución radica en alquilar un recolector para que lo cubra.
\item Se realiza una recogida no selectiva. En la Ordenanza Municipal N$^{\circ}$ 408/14 se establece que los usuarios del servicio ordinario de recolección deberán almacenar sus residuos en el interior de las viviendas o locales correspondientes, en  sitios adecuados, en bolsas perfectamente cerradas, las cuales serán sacadas y colocadas en la acera, solamente en los días señalados, momento antes del horario fijado para el servicio. No se permitirá el acopio o acumulación de los residuos en la vía pública, ya sea directamente sobre las aceras o en canastos elevados en días y horas distintos a los establecidos para el servicio de recolección.
\item El vehículo recolector inicia su recorrido cuando parte de la DSU en dirección a su zona. Antes de partir se controla el nivel de combustible y de acuerdo a esto se procede a la carga del mismo, en caso de que fuese necesario. Se entrega al chofer una orden de trabajo del día con la hora de salida. En la Figura \ref{fig:vehiculoRecolectorMDA}, se muestra un vehículo recolector de la MDA.

\begin{figure}[H]
    \centering
    \includegraphics[width=7cm]{camion_recoletor_mda.png}
    \caption{Vehículo recolector de la MDA. [Fuente: Diario Última Hora]}
    \label{fig:vehiculoRecolectorMDA}
\end{figure}

\item Luego de haber recogido toda la basura domiciliaria de la zona, el vehículo se dirige al relleno sanitario Cateura para depositar los residuos y de allí vuelve a su punto de partida. Generalmente, el vehículo tiene capacidad suficiente para realizar un solo viaje de su zona a Cateura, pero si no es el caso, debe realizar la cantidad de viajes necesarios hasta recolectar todos los residuos domiciliarios de la zona en cuestión. La carga máxima admitida de un vehículo recolector en circulación por el Departamento de Recolección de la DSU es de 8500 Kg, en caso de que el equipo de trabajo estime, en base a su experiencia, que la cantidad total a recoger de la zona será mayor al permitido, entonces realiza más de un viaje a Cateura.
\item Una vez que el chofer llega a Cateura, el camión pasa por un pesaje por ejes con báscula y el chofer entrega la orden de trabajo a un fiscalizador que se encuentra en el puesto de pesaje. Un grupo de personas de seguridad se encargan de coordinar la entrada y salida de vehículos, así como de guiar la descarga de lo recolectado. En el lugar se encuentran varios gancheros que ayudan a sacar los residuos del vehículo, los gancheros son personas dedicadas a recuperar, separar o reducir, de los residuos sólidos urbanos, aquellos que posean valor comercial para su reutilización o reciclaje. Una vez que se vacía el vehículo, uno de los encargados guía nuevamente al chofer hacia la salida, en donde el vehículo es pesado de vuelta, calculando así el peso de lo recolectado, que se detalla en la orden de trabajo la cual es devuelta al chofer. En la orden de trabajo también se especifican la hora de entrada y salida del vehículo al relleno sanitario.
\item Una vez que el equipo de trabajo se encuentra de vuelta en la DSU, el chofer entrega la orden de trabajo al Departamento de Recolección.
\end{itemize}

%TODO: cambiar por el shape que nos dieron%
\begin{figure}[H]
    \centering
    \includegraphics[width=6cm]{Recoleccion-ZONAS&CUADRANTES.png}
    \caption{División de la ciudad en zonas. Un turno en el mapa está correspondido por un conjunto de zonas coloreadas del mismo color. [Fuente: Departamento de Recolección de la DSU]}
    \label{fig:zonasRecoleccion}
\end{figure}

El servicio de recolección domiciliaria se realiza de lunes a sábados, y se divide en tres turnos: mañana, tarde y noche; donde cada turno tiene una duración máxima de 6 horas debido al carácter insalubre de las tareas realizadas por el personal encargado de brindar el servicio.

En la figura \ref{fig:zonasRecoleccion} un turno se corresponde con un conjunto de zonas, las zonas pintadas en azul oscuro, correspondientes al microcentro de Asunción, tienen un tratamiento diferente a las demás zonas, ya que son las únicas que son atendidas de lunes a viernes, las demás zonas reciben el servicio cada dos días, tres veces por semana. Los residuos en estas zonas son recolectados en el turno noche debido al poco tráfico registrado en esas calles en comparación a cualquier otro turno, de igual manera, las avenidas más importantes de la ciudad también son recolectadas en el turno noche de lunes a viernes, los equipos de trabajo que recogen los residuos a lo largo de estas avenidas les corresponden dos avenidas por noche. Las zonas no pintadas no tienen número, y por ende no son cubiertas por el servicio, sin embargo representan una zona candidata a futuro para la recolección de la basura.

Los residuos sólidos de grandes generadores son todos aquellos residuos sólidos urbanos comerciales cuyas cantidades requieran una recolección diferenciada, esta recolección no forma parte del procedimiento descrito más arriba y tampoco se incluye en el alcance de este trabajo. Los residuos hospitalarios son recolectados por una empresa tercerizada por el municipio y no son depositados en Cateura.

% La recolección de los residuos de manejo especial y peligrosos no forman parte lo descrito más arriba y tampoco forman parte del alcance de este trabajo. % Mejora de imagen 

\clearpage
\lhead{\emph{Sistemas de Información Geográfica}} 
\renewcommand\chaptername{Capítulo}%título "Capítulo"
\chapter{Optimización}
\label{chap3}
\ifpdf
    \graphicspath{{Chapter3/Chapter3Figs/PNG/}{Chapter3/Chapter3Figs/PDF/}{Chapter3/Chapter3Figs/}}
\else
    \graphicspath{{Chapter3/Chapter3Figs/EPS/}{Chapter3/Chapter3Figs/}}
\fi

\markboth{\hfill \thechapter. Optimización}{\hfill \thechapter. Optimización}

\section{Proceso de Optimización.}
\label{sec:procOpt}

Consiste en encontrar la mejor solución candidata de entre un conjunto de alternativas. Esta tarea se plantea como un problema estructurado con funciones de variables de decisión, que deben o no satisfacer un conjunto de restricciones.
Un problema de optimización se conforma por una o varias funciones objetivo y (posiblemente) una o varias restricciones.
 \begin{itemize}
     \item \textbf{Variables de decisión:} contienen los valores que se modifican para resolver el problema.
     \item \textbf{Función o funciones objetivo:} Estas funciones se expresan en términos de las variables de decisión, y el resultado de su evaluación es el que se desea optimizar (maximizar o minimizar). Si solo una función es considerada se habla de un problema de optimización mono-objetivo. Si varias funciones son consideradas, el problema se denomina optimización multi-objetivo.
     \item \textbf{Restricciones:} expresadas en forma de ecuaciones de igualdad o desigualdad, se deben cumplir o satisfacer para que la solución sea considerada factible, es decir, válida. Si el problema no presenta restricciones, todas las soluciones son válidas.
 \end{itemize}


\section{Optimización Multiobjetivo.}
\label{sec:multiobj}

% La mayoría de los problemas y situaciones en la vida real son reconocidos como 
Problemas multiobjetivo implican la optimización simultánea de dos o más objetivos, y no poseen un único criterio medible por el cual pueda decirse que una solución sea completamente satisfactoria. En otras palabras, este tipo de problemas contiene múltiples objetivos que han de satisfacerse o que han de ser tenidos en cuenta. A menudo dichos objetivos entran en conflicto unos con otros y no existe una única solución que simultáneamente satisfaga a todos.  Por eso, la solución que se pretenda obtener queda exclusivamente a cargo del tomador de decisiones. Las soluciones de este tipo de problemas se consideran óptimas porque ninguna otra solución es superior a ellas cuando se tiene en cuenta todos los objetivos a la vez \cite{nesmachnow2004version}.

El conjunto de soluciones donde una función objetivo no puede ser mejorada sin empeorar alguno de los otros objetivos se llama \textit{Conjunto Pareto Óptimo}. El Conjunto Pareto Óptimo se define como \cite{nesmachnow2004version}:

\begin{equation}\label{pareto_optimo}
    P^* = \left\{{{x}\in{ \Omega} | \urcorner \exists{{x^{\prime}}}\in{ \Omega}, {f}({x^{\prime}}) \preceq {f}({x})}\right\}
\end{equation}

Donde:\\
$\Omega = \left \{ \vec{x} \in  \mathbb{R}^n \right \}$: es la región de soluciones factibles, y cualquier punto en ${x}\in{\Omega}$ es una solución factible.
% $\Omega$ El espacio de búsqueda o conjunto de todas las soluciones posibles.
% $x_i \in \Omega$ un elemento del conjunto de posibles soluciones


El conjunto de soluciones que se encuentran en la región de soluciones factibles son soluciones no dominadas o también llamados conjunto no-dominado o \textit{Frontera Pareto}, que se define como:

\begin{equation}\label{frente_pareto}
     FP^* = \left\{ {f_1({x}), f_2({x}),  f_3({x}), \ldots,  f_N({x}) | {x}\in{P^*} }\right\}
\end{equation}

mientras que las soluciones que se encuentran dentro de la región factible son las llamadas soluciones dominadas, porque siempre hay otra solución en la región que tiene cuando menos un objetivo mejor.

Los métodos para encontrar la mejor solución (óptima) varían de acuerdo con la complejidad del problema enfrentado. Para problemas triviales, el cerebro humano posee la capacidad de resolverlos (encontrar la mejor solución) directamente, pero a medida que el problema es más complejo, es necesario buscar herramientas adicionales.

Existen métodos que vienen del área de Inteligencia Artificial que consisten en sistematizar ideas para desarrollar algoritmos eficientes que encuentran \textbf {buenas soluciones} a problemas de optimización; estas soluciones, en muchos casos son aproximadas a la solución óptima. Las técnicas metaheurísticas son útiles cuando se desean resolver problemas cuyo modelo matemático no puede ser formulado fácilmente o cuando tienen espacios de búsqueda muy grandes. Las mismas combinan la simplicidad de sus ideas con su gran eficiencia para obtener muy buenas soluciones para este tipo de problemas \cite{cuartasmetodologia}. 

Las más conocidas son los algoritmos evolutivos, colonia de hormigas, enjambre de partículas y enfriamiento simulado \cite{lima2007}. 

El objetivo de los métodos de optimización multiobjetivo es encontrar el conjunto de soluciones no dominadas y no una solución única.

\section{Optimización Robusta.}
\label{sec:robusta}

Para los problemas de optimización del mundo real, el \textbf{entorno de decisión} a menudo se caracteriza por los siguientes hechos \cite{BenTal2002RobustO}:
\begin{itemize}
    \item \textbf{F.1.} Los datos son inciertos/inexactos;
    \item \textbf{F.2.} La solución óptima, incluso si se calcula con mucha precisión, puede ser difícil de implementar con precisión;
    \item \textbf{F.3.} Las restricciones deben seguir siendo factibles para todas las realizaciones significativas de los datos;
    \item \textbf{F.4.} Los problemas son a gran escala;
    \item \textbf{F.5.} Las soluciones óptimas \textbf{malas} (aquellas inviables incluso con cambios relativamente pequeños en los datos nominales) no son infrecuentes.
 \end{itemize}
 
\textit{La Optimización Robusta} ($RO$, por sus siglas en inglés, Robust Optimization) es una metodología de modelado, combinada con un conjunto de herramientas computacionales, que tiene como objetivo cumplir con los hechos anteriores. La urgencia de contar con tal metodología proviene del hecho \textbf{F.5} \cite{BenTal2002RobustO}.

En la metodología de Optimización robusta, uno se asocia con un problema incierto, su contraparte robusta, que es un programa de optimización usual (semi-infinito) \cite{BenTal2002RobustO}.

% La Optimización Robusta es un enfoque relativamente nuevo para modelar la incertidumbre en los problemas de optimización. Mientras que la programación estocástica asume que hay una descripción probabilística de la incertidumbre, la optimización robusta funciona con una descripción determinista basada en conjuntos de la incertidumbre. El enfoque de optimización robusto construye una solución que es factible para cualquier realización de la incertidumbre en un conjunto dado.

% La Optimización Robusta es un subcampo importante de la optimización que se ocupa de la incertidumbre en los datos de los problemas de optimización. Bajo este contexto, se supone que las funciones objetivo y de restricción solo pertenecen a ciertos conjuntos en el espacio funcional (los denominados \textit{conjuntos de incertidumbre}). El objetivo es tomar una decisión que sea factible sin importar cuáles sean las restricciones, y el óptimo para la función objetivo del peor de los casos.

La formula general de la Optimización Robusta es:
\begin{equation} \label{ro}
\begin{split}
& \textrm{ minimizar} \quad f_{0}(x) \\
& \textrm{sujeto} \quad \textrm{a} \quad f_{i}(x,u_{i}) \leq 0 \quad \forall u_{i} \in U_{i}, \quad \textrm{i=1...m}
\end{split}
\end{equation}

% objetivo a optimizar f0(x)
% restricción fi(x,ui) ≤ 0 
% parametros inciertos {ui}, los datos que son parte de las entradas del problema de optimización.

Donde:
\begin{itemize}
     \item $x \in \mathbb{R}^{n}$, es un vector de variables de decisión.
     \item $f_{0}$, $f_{i}$: \quad$\mathbb{R}^{n} \rightarrow \mathbb{R}$, son funciones.
     \item $u_{i} \in \mathbb{R}^{k}$, son parametros de incertidumbre, toman valores arbitrarios en los conjuntos de incertidumbre $U_{i} \subseteq \mathbb{R}^{k}$.
 \end{itemize}
 
%  El objetivo de {ro} (ecuación), es calcular soluciones de costo minimo x, entre todas las soluciones que son factibles para todas las realizaciones de las perturbaciones ui dentro de Ui. 



\section{Optimización Multiobjetivo Robusta.}
\label{sec:multiRo}

Para una optimización de un solo objetivo, una solución robusta se define como la que es insensible (hasta un límite) a la perturbación en las variables de decisión en su vecindad \cite{Deb2006IntroducingRI}.

Varios investigadores han sugerido diferentes procedimientos para definir y encontrar soluciones robustas en un contexto de optimización de un solo objetivo. Una de las principales ideas retratadas en la literatura es utilizar una función objetivo efectiva media para la optimización, en lugar de la función objetivo en sí misma \cite{Deb2006IntroducingRI}.

Un problema de optimización multiobjetivo tiene varios objetivos en conflicto:

\begin{equation} \label{ro_multi}
\begin{split}
& \textrm{ minimizar} \quad \bigl(\begin{smallmatrix}
 f_{1}(x), & f_{2}(x), & ..., & f_{m}(x)
\end{smallmatrix}\bigr) \\
& \textrm{sujeto} \quad \textrm{a} \quad  x \in S
\end{split}
\end{equation}

El objetivo en una optimización evolutiva multiobjetivo es encontrar un número finito de soluciones óptimas de Pareto, en lugar de un único óptimo. Dado que las soluciones óptimas de Pareto dominan colectivamente cualquier otra solución factible en el espacio de búsqueda, todas se consideran mejores que cualquier otra solución. Se dice que una solución domina a otra solución si no es peor en ninguno de los objetivos y es estrictamente mejor en al menos uno de los objetivos. Para calificar como una solución robusta, cada solución óptima de Pareto ahora tiene que demostrar su insensibilidad frente a pequeñas perturbaciones en sus valores de variables de decisión \cite{Deb2006IntroducingRI}.

Las principales diferencias con una solución robusta de un solo objetivo son las siguientes \cite{Deb2006IntroducingRI}:
\begin{enumerate}
     \item La sensibilidad ahora tiene que establecerse con respecto a todos los objetivos (o a los preferidos por el tomador de decisiones). Es decir, se debe usar un efecto combinado de variaciones en todos los objetivos como una medida de sensibilidad a la perturbación variable.
     \item Hay muchas soluciones para verificar la robustez, en lugar de una o dos soluciones como en el caso de la optimización de un solo objetivo.
\end{enumerate}

\subsection{Optimización de Enjambre de Partículas.}
\label{sec:pso}
% porque pso y no otro algoritmo (porq pso ayuda a representar mejor el problema, más asociado a la naturaleza)

La \textit{Optimización por Enjambres de Partículas} (conocida como $PSO$, por sus siglas en inglés, Particle Swarm Optimization) es una metaheurística basada en poblaciones e inspirada en el comportamiento social del vuelo de las bandadas de aves o el movimiento de los bancos de peces, desarrollado por Jammes Kennedy y Russell Eberhart \cite{swarmintelligence}.

% Cada solución (partícula) es un \textbf{ave} en el espacio de búsqueda que está siempre en continuo movimiento y que nunca muere. El cúmulo de partículas (swarm) es un sistema multiagente, es decir, las partículas son agentes simples que se mueven por el espacio de búsqueda y guardan la mejor solución que han encontrado. Cada partícula tiene un $fitness$ (función de aptitud), una $posición$ y un \textit{vector velocidad} que dirige su \textbf{movimiento}. El movimiento de las partículas por el espacio está guiado por las partículas óptimas en el momento actual.

Los algoritmos basados en cúmulos de partículas se han aplicado con éxito en diferentes campos de investigación; entrenamiento de redes neuronales \cite{neuralnetworks}, aprendizaje de sistemas difusos \cite{fuzzyswarmoptimization}, registrado de imágenes \cite{imageswarmoptimization}.

El algoritmo $PSO$ consiste en un proceso iterativo y estocástico que opera sobre un cúmulo de partículas. La posición de cada partícula representa una solución potencial al problema que se está resolviendo. Generalmente, una partícula $\rho_i$ está compuesta de tres vectores y dos valores de fitness:
\begin{itemize}
    \item \textbf{El vector $\chi_i$} = ($\chi_{i1}, \chi_{i2}, ..., \chi_{in})$ almacena la posición actual (localización) de la partícula en el espacio de búsqueda.
    \item \textbf{El vector $\rho{Best_i}$} = ($\rho{Best}_{i1}, \rho{Best}_{i2}, ..., \rho{Best}_{in})$ almacena la posición de la mejor solución encontrada por la partícula hasta el momento.
    \item \textbf{El vector de velocidad $v_i$} = ($v_{i1}, v_{i2}, ..., v_{in})$ almacena el gradiente (dirección) según el cual se moverá la partícula.
    \item \textbf{El valor de $fitness_{\chi_i}$} almacena el valor de adecuación de la solución actual (vector $\chi_i$).
    \item \textbf{El valor de $fitness_{\rho{Best_i}}$} almacena el valor de adecuación de la mejor solución local encontrada hasta el momento (vector $\rho{Best_i}$).
\end{itemize}

El cúmulo se inicializa generando las posiciones y las velocidades iniciales de las partículas. Las posiciones se pueden generar aleatoriamente en el espacio de búsqueda. Una vez generadas las posiciones, se calcula la aptitud de cada partícula y se actualizan los valores de $fitness_{\chi_i}$ y $fitness_{\rho{Best_i}}$.

Las velocidades se generan aleatoriamente, con cada componente en el intervalo $[-v_{max}, v_{max}]$, donde $v_{max}$ será la velocidad máxima que pueda tomar una partícula en cada movimiento. No es conveniente dejarlas a cero pues no se obtienen buenos resultados \cite{swarmintelligence}.

El vector velocidad de cada partícula $\rho_i$ se actualiza en cada iteración y se calcula según la siguiente formula:

\begin{equation}\label{velocidad}
    v^{t+1}_i = \omega \times v^{t}_i + \varphi_1 \times rand_1 \times (\rho{Best_i} - \chi^{t}_i) + \varphi_2 \times rand_2 \times (g_i - \chi^{t}_i)
\end{equation}

Donde:
\begin{itemize} 
    \item $t$: iteración, donde $t$ = 1,2,3..,n
    \item $v^{t}_i$: velocidad de la partícula $\rho_i$ en la iteración $t$;
    \item $\omega$ :  coeficiente de inercia;
    %specific parameters which control the effect of the personal and global best particles.
    \item $\varphi_1, \varphi_2$ : parámetros que controlan los componentes cognitivo y social de las partículas;
    %\item $\varphi_1, \varphi_2$ : factores de aprendizaje (pesos) que controlan los componentes cognitivo y social;
    \item $rand_1, rand_2$ : números aleatorios entre 0 y 1.;
    \item $\chi^{t}_i$: posición actual de la partícula $\rho_i$ en la iteración $t$;
    \item $\rho{Best_i}$: mejor posición (solución) encontrada por la partícula $i$ hasta el momento;
    \item $g_i$: representa la posición de la partícula con el mejor $fitness_{\rho{Best_i}}$ del entorno de $\rho_i$ ($lBest$ o $localbest$) o de todo el cúmulo ($gBest$ o $globalbest$);
    \item $\varphi_1 \times rand_1 \times (\rho{Best_i} - \chi^{t}_i)$: componente cognitivo, representa la distancia entre la posición actual y la mejor conocida por esa partícula, la decisión que tomará la partícula influenciada por su propia experiencia a lo largo de su vida;
    \item $\varphi_2 \times rand_2 \times (g_i - \chi^{t}_i)$: componente social, representa la distancia entre la posición actual y la mejor posición del vecindario, es decir, la decisión que tomará la partícula según la influencia que el resto del cúmulo ejerce sobre ella.
\end{itemize}

La particula $\rho_i$ actualiza su posición $\chi$ de acuerdo a la siguiente ecuación:
\begin{equation}\label{posicion}
    \chi^{t+1}_i = \chi^{t}_i + v^{t+1}_i
\end{equation}

El pseudocódigo del $PSO$ básico se presenta en el Algoritmo \ref{pso_basico}.

\begin{algorithm}[hbpt]
    \begin{algorithmic}[1]
    \REQUIRE número de partículas $\Omega$, número de iteraciones $t$: Se recibe como parámetros el número de partículas $\Omega$, y el número de iteraciones $t$.
    \STATE Inicializar la posición $\chi_i$ y velocidad $v$ de cada partícula aleatoriamente en cada dimensión $D$
    \STATE Inicializar los mejores individuales $\rho{Best_i}$ de cada partícula con su posición inicial $\chi_i$.
    \STATE Inicializar el mejor global $g_i$
    \STATE Inicializar el parámetro $\omega$
    \WHILE {no se cumpla criterio de parada o no se llegue al máximo de iteraciones $t$}
        \FOR {cada $i$-ésima partícula del enjambre}
            \STATE Evaluar la función de aptitud (fitness) de la partícula $\rho_i$ en la iteración $t$
            \IF {el fitness de la partícula es mejor que el fitness del mejor individual $\rho{Best_i}$}
                \STATE reemplazar $\rho_i$ por el nuevo valor de $\chi^t_i$
            \ENDIF
             \IF {el fitness de la partícula es mejor que el fitness del mejor global $g_i$}
                \STATE reemplazar $g_i$ por el nuevo valor de $\chi^t_i$
            \ENDIF
            \STATE Calcular la nueva velocidad de la partícula $v^{t+1}_i$
            \STATE Calcular la nueva posición de la partícula $\chi^{t+1}_i$
        \ENDFOR
    \ENDWHILE
    \RETURN mejor global $g_i$: retorna mejor solución encontrada.
    \end{algorithmic}
    \caption{Algoritmo básico del \textit{PSO}.}
    \label{pso_basico}
\end{algorithm} \break

\subsection{Speed-constrained Multi-objective PSO (SMPSO)}
\label{sec:smpso}

El \textit{SMPSO} es una metaheurística basada en el algoritmo \textit{OMOPSO}, un optimizador de enjambre de partículas multiobjetivo (\textit{MOPSO}) (por sus siglas en inglés, Multi Objective Particle Swarm Optimizer) que fue diseñado para tratar con problemas de optimización multiobjetivo \cite{RC05}. 

Se basa en un conjunto de partículas, llamado población y contiene un repositorio global en el que cada partícula deposita su experiencia por cada iteración, generando un archivo con posiciones dominantes, actualizado en cada iteración para generar el conjunto Pareto. El enfoque \textit{SMPSO} aplica un esquema de limitación de velocidad para expandir la capacidad de exploración, así como mejorar la rapidez de convergencia. La velocidad de las partículas es limitada, en lugar de utilizar los parámetros máximo y mínimo para limitar el tamaño del cambio de la velocidad.  \cite{daumasjara}

El \textit{SMPSO} incorpora un mecanismo de restricción (ecuación \ref{restriccion}) que se obtiene del factor de restricción $\kappa$ desarrollado por Clerk y Kennedy sobre la (ecuación \ref{velocidad}) para limitar la velocidad máxima de las partículas y mejorar la capacidad de búsqueda del algoritmo \cite{smpso}.

\begin{equation}\label{restriccion}
    \kappa   = \frac{2}{2 - \sigma - \sqrt[]{ \sigma^2 - 4 \sigma}}
\end{equation}

Donde:
\begin{equation}\label{phi}
 \sigma = \left\{{\begin{tabular}{cc}
  $\varphi_1 +  \varphi_2$ & si $\varphi_1 +  \varphi_2 > 4$ \\
  $0$         & si $\varphi_1 +  \varphi_2 \leq 4$
  \end{tabular}}\right\}
\end{equation}

Además, se introduce un mecanismo de tal manera que la velocidad acumulada de cada variable $j$ (en cada partícula $\rho_i$) esté limitada por medio de la siguiente ecuación de restricción de velocidad:

\begin{equation}\label{velocidad_limitada}
 v^{t+1}_{i,j} = \left\{{\begin{tabular}{cc}
  $\delta_j$ & si $ v^{t+1}_{i,j} > \delta_j$ \\
  $-\delta_j$ & si $ v^{t+1}_{i,j} \leq - \delta_j$ \\
  $ v^{t+1}_{i,j}$ & para otros casos
  \end{tabular}}\right\}
\end{equation}

Donde:

\begin{equation}\label{delta}
\delta_j = \frac{maximo_j - minimo_j}{2}
\end{equation}

para:\\
$maximo_j$ = máximo valor para la variable $j$ a optimizar.\\
$minimo_j$ = mínimo valor que puede tomar la variable  $j$.

Con el enfoque \textit{SMPSO}, la velocidad de las partículas $\rho_i$ se calcula de acuerdo con la (ecuación \ref{velocidad}); la velocidad resultante se multiplica por el factor de restricción de la (ecuación \ref{restriccion}) y el valor resultante se limita usando la (ecuación \ref{velocidad_limitada}).

El archivo de líderes es actualizado insertando las partículas no dominadas que existen, eliminando las dominadas en el proceso. El tamaño del archivo de líderes es limitado, debido a esto, cuando se llena, \textit{SMPSO} aplica el criterio de distancia de hacinamiento (crowding), especificado en el algoritmo NSGA-II \cite{NSGA} y es empleada como operador para mantener la diversidad en las partículas. Esta es calculada para cada partícula sumando las distancias entre los individuos inmediatamente mayor y menor considerando cada objetivo a evaluar. Así, los individuos con mayor distancia de hacinamiento (amontonamiento) son asignados al archivo de líderes a diferencia de aquellos con menor distancia de hacinamiento \cite{daumasjara}.
El operador de turbulencia utilizado por \textit{SMPSO} es la mutación polinomial \cite{polinomial}.

El pseudocódigo del $SMPSO$ se presenta en el Algoritmo  \ref{smpso_algorithmic}.

%\begin{algorithm}
%    \begin{algorithmic}[1]
%    \STATE initializeSwarm()
%    \STATE initializeLeadersArchive()
%    \STATE generation = 0
%    \WHILE {generation < maxGenerations}
%        \STATE computeSpeed(),  \ref{velocidad}, \ref{restriccion}, \ref{velocidad_limitada}
%        \STATE updatePosition(), \ref{posicion}
%        \STATE mutation() // Turbulence
%        \STATE evaluation()
%        \STATE updateLeadersArchive()
%        \STATE updateParticlesMemory()
%        \STATE generation ++
%    \ENDWHILE
%    \RETURN LeadersArchive
%    \end{algorithmic}
%    \caption{Algoritmo del $SMPSO$.}
%    \label{smpso_algorithmic}
%\end{algorithm} 
%\break 

\begin{algorithm}[hbpt]
    \begin{algorithmic}[1]
    \STATE inicializarEnjambre(): Se inicializa el enjambre, que incluye la posición, la velocidad y la mejor posición individual de las partículas.
    \STATE inicializarArchivoLideres(): Se inicializa el archivo de líderes con las soluciones no dominadas en el enjambre.
    \STATE generacion = 0
    \WHILE {generacion $<$ maxGeneraciones}
        \STATE calcularVelocidad(): Se calcula la velocidad de cada partícula
        \STATE actualizarPosicion() : Se calcula y actualiza la posición de cada partícula
        \STATE mutacion(): Se aplica un operador de mutación con una probabilidad dada.
        \STATE evaluacion(): Se evalúan las partículas resultantes.
        \STATE actualizarArchivoLideres()
        \STATE actualizarParticulas()
        \STATE generacion ++
    \ENDWHILE
    \RETURN ArchivoLideres: El algoritmo devuelve el archivo de líderes como el conjunto de aproximación encontrado.
    \end{algorithmic}
    \caption{Algoritmo del \textit{SMPSO}.}
    \label{smpso_algorithmic}
\end{algorithm}  % Optimización PSO

\clearpage
\lhead{\emph{Optimización}} 
\renewcommand\chaptername{Capítulo}%título "Capítulo"
\chapter{Optimización}
\label{chap4}
\ifpdf
  \graphicspath{{Chapter4/Chapter4Figs/PNG/}{Chapter4/Chapter4Figs/PDF/}{Chapter4/Chapter4Figs/}}
\else
  \graphicspath{{Chapter4/Chapter4Figs/EPS/}{Chapter4/Chapter4Figs/}}
\fi

\markboth{\hfill \thechapter. Optimización}{\hfill \thechapter. Optimización}


\citet{Akhtar2017BacktrackingOptimization} proponen combinar un algoritmo metaheurístico con contenedores de basura inteligente equipados con diferentes sensores, el estudio introduce el concepto del umbral de nivel de residuos (TWL, \textit{Threshold Waste Level}) de los contenedores de residuos para reducir el número de contenedores que se deben vaciar, al encontrar un rango óptimo, minimizando así la distancia.

\citet{Kallel2016UsingTunisia} desarrolló escenarios optimizados utilizando la herramienta ArcGIS Network Analyst para mejorar la eficiencia de la recolección de residuos y el transporte en el distrito Cívico El Habib de la ciudad de Sfax, Túnez.

 % Métricas de evaluación

\clearpage
\lhead{\emph{Propuesta}} 
\renewcommand\chaptername{Capítulo}%título "Capítulo"
\chapter{Modelado del problema}
\label{chap5}
\ifpdf
  \graphicspath{{Chapter5/Chapter5Figs/PNG/}{Chapter5/Chapter5Figs/PDF/}{Chapter5/Chapter5Figs/}}
\else
  \graphicspath{{Chapter5/Chapter5Figs/EPS/}{Chapter5/Chapter5Figs/}}
\fi

\markboth{\hfill \thechapter. Modelado del problema}{\hfill \thechapter. Modelado del problema}

\section{Planteamiento del problema}
% \label{cap:planteamiento}

La fase de recolección de residuos sólidos cumple un rol importante en los aspectos socio-económicos y ambientales de una ciudad. Según la literatura, gran parte del presupuesto de un municipio va destinada a dicha fase. En consecuencia, se genera la necesidad de una búsqueda permanente por disminuir costos en sus procesos sin afectar la calidad del servicio. 

Hoy en día, la selección de la ruta de recolección se basa en la propia intuición y experiencia de los conductores. En algunas ocasiones esto conlleva a dejar sin servicio algunos puntos de la ciudad por desconocimiento de un nuevo conductor asignado a la zona, que a su vez genera que la comuna asuncena realice quejas acerca de la falta de servicio de recolección en tiempo y forma. También es posible el paso en repetidas veces de forma innecesaria por la misma calle dando lugar a mayores gastos de combustible, como se indica en los círculos rojos de la Figura \ref{fig:trayectoRecoleccion}.

\begin{figure}[tb]
    \centering
    \includegraphics[width=14.5cm]{20170329_recorrido_repetido.png}
    \caption{Trayecto del vehículo 62 en fecha 11 de Julio del 2016 en el turno mañana. [Fuente: Datos de rastreo vía GPS desplegados en la aplicación QGis]}
    \label{fig:trayectoRecoleccion}
\end{figure}

Otra situación muy común es el frecuente cambio de circulación en las calles buscando disminuir la congestión del tráfico actual, por ejemplo: cambio de sentido, prohibición de giro a la izquierda, prohibición de giro en U, contramano. También se presentan situaciones donde las calles quedan clausuradas para su uso por motivo de reparación de la capa asfáltica, trabajos de instalación o reparación de cañería.

Por ello se evidencia la necesidad de una herramienta de soporte de decisión que aborde el problema del enrutamiento de los vehículos recolectores teniendo en cuenta el procedimiento actual llevado a cabo para la recolección de residuos domiciliarios de la DSU.

% Para  ello  se  implementa  una  solución  GIS  que  optimice  el  camino  del  vehículo recolector teniendo en cuenta el procedimiento actual llevado a cabo para la recolección de residuos domiciliarios, gestionando de forma sencilla y rápida el estado actual de las calles y sus reglas de circulación.

% En la Figura \ref{fig:trayectoRecoleccion}, se muestra una pequeña parte del rastreo de una zona, en los círculos de color rojo se pueden observar como el mismo vehículo recorre la misma calle más de una vez. Esta situación se pudo contemplar en los recorridos de varias zonas. Para la recolección de datos se solicitó a la DSU el permiso de instalar un dispositivo GPS a un vehículo recolector de basura, como parte de este proyecto, a través del cual obtuvimos por un periodo de 3 meses los datos de posicionamiento relacionados al vehículo 62 de propiedad de la Municipalidad de Asunción. Esta idea ya generó el interés de la Municipalidad de Asunción, que posteriormente ha realizado una licitación para dotar a todos los vehículos recolectores de residuos un dispositivo similar GPS, brindándonos acceso a los datos del rastreo de todos vehículos recolectores.

\section{Metodología de trabajo}

En la Figura \ref{fig:metodologia} se muestra la metodología seguida en este trabajo: la recolección de datos, selección de un modelo matemático adecuado para resolver el problema de ruteo, la implementación de la herramienta \textit{TapeYty}, el despliegue de la ruta óptima y por último el análisis de los resultados.

\begin{figure}[htbp]
\centerline{\includegraphics[width=\textwidth]{DiagramaDeMetodologia.png}}
\caption{Vista general de los procesos de la metodología aplicada.}
\label{fig:metodologia}
\end{figure}

% La solución del trabajo de investigación es lo que se buscará encontrar en el transcurso del desarrollo del TFG. La solución encontrada debe permitir obtener mejores resultados en cuanto a costo de recolección, en comparación a la situación actual en la DSU del municipio de Asunción, así como también, generar resultados óptimos que contribuyan con el estado del arte actual

\section{Colección de Datos}

A modo de desplegar las rutas de recolección en el mapa de la aplicación se debe contar necesariamente con un mapa de calles y otro de zonas definidas por la DSU de recolección. Tanto el mapa de zonas como el de calles deben estar almacenados en la base de datos GIS para ser posteriormente utilizados durante el procesamiento de \textit{TapeYty}. Se crea una base de datos espacial utilizando el gestor de base de datos de código abierto PostgreSQL en su versión 10.8 \citep{PostgreSQL}, junto con su extensión PostGIS en su versión 2.4 \citep{PostGIS}, la cual brinda soporte para objetos geográficos.

Primeramente, se necesitó exportar los datos de Asunción desde OSM, debido a que no se tuvo conocimiento de alguna institución que proveyera de mapas de calles de la ciudad con sus sentidos. A continuación se detallan los pasos seguidos para obtener la red de rutas utilizada: 
\begin{enumerate}
    \item Se utiliza la herramienta \textit{osm2pgsql} para importar los datos de OSM a la base de datos, en el sistema de referencia geográfico WGS84 (SRID 4326). 
    \item Se utilizan las funciones de PostGIS para convertir los valores de latitud y longitud al tipo de dato geométrico \textit{Point}. 
    \item Las calles de OSM están representadas por líneas, que a su vez contienen una serie de nodos (puntos). Se excluyen los puntos que no pertenecen a calles de la ciudad, además de puntos superpuestos.
    \item Para un mejor manejo de la red de rutas se crea un proceso de segmentación de las calles, donde cada segmento de calle está formado por pares de nodos consecutivos, se define el sentido del segmento, longitud y si corresponde o no a un callejón sin salida.
    \item Por último, se realizan las transformaciones correspondientes para almacenar las restricciones de giro y contramano, desde los datos registrados en OSM.
\end{enumerate}

Para almacenar la información de red de calles se definen las siguientes tablas en la base de datos GIS:

\begin{itemize}
    \item \textbf{Nodo:} En esta tabla se almacenan los puntos que se utilizan como extremos de los segmentos de calle. Se almacena la latitud y la longitud.
    \item \textbf{Segmento:} Contiene líneas de calle formada a partir de dos nodos consecutivos. Se almacena el nombre, el nodo origen, el nodo destino, si es de único de sentido y si es sin salida. Si el segmento es de único sentido, entonces el nodo origen y destino definen el sentido del mismo.
\end{itemize}

Al importar los datos desde OSM se detectaron algunos datos incorrectos que generarían inconvenientes para la obtención del recorrido para la recolección de residuos. OSM identifica las calles sin salida por medio del atributo \textit{no\_exit}. Si su valor es \textit{true} representa una calle sin salida, sin embargo muchas calles sin salida lo tienen como \textit{false}. Otra inconsistencia encontrada es con el valor del atributo \textit{one\_way}, donde muchas calles sin salida tienen su valor en \textit{true} indicando que es de único sentido cuando realmente deben ser doble sentido.

Los problemas de datos mencionados se solucionan con sentencias SQL. Para identificar las calles sin salida se realiza una consulta donde se obtenga la unión de: ``todos los segmentos que tengan un nodo inicial que no sea nodo final ni inicial de algún otro segmento'' y ``todos los segmentos que tengan un nodo final que no sea inicial o final de otro segmento''. Se obtiene como resultado la lista de todos los ID (identificadores) de los segmentos sin salida y se almacena en una tabla temporal. Se actualizan los campos de sin salida y sentido de todos aquellos segmentos cuyos ID se encuentran en la tabla temporal.

La delimitación de las zonas de recolección fue proveída por la DSU a través de un archivo \textit{shape} del tipo de dato geométrico \textit{Polygon}. Se realizan los siguientes procedimientos para la limpieza y corrección de los datos geográficos sobre dicha capa:

\begin{enumerate}
\item Se importa el \textit{shape} de Zonas a la base de datos mediante la herramienta \textit{shp2pgsql}, creándose así la tabla de zonas.
\item Se utiliza la opción de autoensamblado de QGIS. Esta opción ayuda a hacer coincidir con precisión los límites de las zonas con las calles de OSM.
% \item Con el programa QGIS, se agregan las capas PostGis de Segmentos, Nodos (de los segmentos) y la nueva de Zonas.
% \item Se configura la opción de autoensamblado de QGIS. Esta opción ayuda para hacer coincidir con precisión la edición de capas con los nodos de los segmentos. Es decir, se utilizan los nodos de los segmentos para hacer coincidir con los nodos de las líneas que formarán el polígono de la zona.
\item En la tabla de zonas se almacenan: nombre, superficie, densidad poblacional, cantidad de lotes y distrito.
\end{enumerate}

Para actualizar la superficie se utiliza la función geométrica de área de polígonos. Para actualizar la cantidad de lotes y distrito se utiliza la función de intersección. No se cuentan con datos de densidad poblacional de todas las zonas de recolección.

Los mapas bases de Departamentos y Distritos son proveídos a la aplicación por el Servicio Nacional de Catastro (SNC) a través de su portal de datos abiertos.

La DSU cuenta en algunos vehículos de su flota con un sistema de monitoreo mediante GPS: 

\begin{itemize}
    \item Se accede a los datos a través de archivos en formato de plantilla.
    \item Esta información es utilizada para el análisis y comparación de los resultados.
\end{itemize}

\section{Modelo matemático}

Como resultado de la revisión del estado del arte, se utiliza la solución propuesta por \citet{Braier2017AnArgentina} para optimizar el enrutamiento de vehículos recolectores, ya que este modelo contempla las restricciones de la red de rutas, resolviendo una de las debilidades de los enfoques de la programación matemática mencionado en \cite{Sulemana2018OptimalMethods}. Además, la falta de datos relacionados con la cantidad de toneladas recolectadas por zonas fue otro de los principales motivos por el cual se seleccionó un problema de ruta cuya demanda se encuentra sobre los arcos (calles que deben ser visitadas), y además no posea restricciones de capacidad.

El estudio en \cite{Braier2017AnArgentina} presenta similitudes con el caso de estudio de este trabajo, entre ellas es posible citar:

\begin{itemize}
    \item División de la ciudad en sectores y recolección casa por casa: La situación es muy similar, ya que el procedimiento consiste en recoger los residuos domiciliarios casa por casa, debiendo cubrir todas las calles de un conjunto de cuadras o manzanas, denominadas zonas.
    \item Tamaño de problema: Una zona de recolección abarca en promedio el mismo número (entre 40 y 80 cuadras aproximadamente).
    \item Calles, carreteras, caminos: el trabajo de \citet{Braier2017AnArgentina} tomó como caso de estudio la ciudad de Morón el cuál presenta caminos con particularidades muy comunes a otras sudamericanas como la de Asunción: calles sin salida, calles estrechas, peatonales, giros prohibidos, entre otros.
\end{itemize}

Los siguientes supuestos son establecidos:

\begin{itemize}
    \item El vehículo recolector cuenta con capacidad suficiente para recoger los residuos de una zona determinada.
    \item El tráfico es constante en una zona de trabajo en el turno en que se recogen sus residuos.
    \item Las modificaciones geográficas de los datos espaciales estarán a cargo de una persona capacitada en el área GIS.
    % La persona encargada de la administración del sistema posee conocimientos básicos en GIS y podrá actualizar, crear y eliminar una zona de trabajo mediante el sistema \textit{QGis}
\end{itemize}

Cada zona de recolección de la ciudad de Asunción es representada por un grafo mixto $H$ \citep{Braier2017AnArgentina}, cuyos nodos representan las esquinas de las calles en la zona, y los arcos son los segmentos de calles que corren entre dos intersecciones consecutivas. Las calles de un único sentido están representadas por arcos dirigidos y las calles finas de doble sentido por arcos no dirigidos, ya que ambos lados de la calle pueden ser servidos en un solo viaje. En el caso de calles anchas de doble sentido, como las avenidas, cada lado debe ser servido de forma separada, por lo que estas calles se representan con dos arcos dirigidos, una en cada sentido.

Para incorporar las restricciones de regulación de tráfico se construye un grafo dirigido $G'$ desde el grafo $H$. El grafo $H$ es expandido dividiendo cada nodo en varios nuevos nodos representando todas las formas en las que se puede llegar y salir de la esquina en cuestión. En la Figura \ref{fig:grafo_expandido}(a) se puede observar un nodo en el grafo original $H$ antes de su expansión y en la Figura \ref{fig:grafo_expandido}(b) el nodo que ha sido expandido en seis nuevos nodos, representando cada posible entrada y salida del nodo. Los arcos auxiliares dirigidos son agregados y conectan los nuevos nodos, representando así las transiciones permitidas de una esquina a otra.

El modelo de programación entera propuesto por \citet{Braier2017AnArgentina} está definido de la siguiente manera:

\subsection{Conjuntos y Parámetros}
\label{sec:conjunto-parametros}

\begin{itemize}
\item $G'(V, A)$: Grafo dirigido en el que los nodos V corresponden a todas las alternativas posibles para llegar a las esquinas de las intersecciones y A está compuesto por arcos que se pueden atravesar en una sola dirección específica.

\item $E \subseteq \{ \{i, j\}: i \in V, j \in V, i \neq j\}$: Representan segmentos de calles de dos vías que pueden ser recorridos en cualquier sentido.

\item $AM \subseteq A $: Arcos obligatorios que representan segmentos que deben recorrerse únicamente en el sentido especificado.

\item $w : A \rightarrow \mathbb{R} $: Función de peso que asocia un peso a cada arco, en este caso la distancia del segmento de calle. Los arcos auxiliares entre nodos que representan esquinas tienen un mismo valor ínfimo.

\item $I \subseteq V $: Nodos que especifican los puntos de inicio permitidos para la ruta.

\item $\mathcal{S} \subseteq V$: Se define $\delta^+ (\mathcal{S}) = \{i j \in A: i \in S , j \notin \mathcal{S} \}$ , que representa un conjunto de arcos que van desde nodos en $\mathcal{\mathcal{S}}$ a nodos en $V \backslash \mathcal{S}$.
\end{itemize}

El problema de ruteo es una versión particular del problema del cartero rural abierto dirigido generalizado, ya que el arco $i j \in E $ determina el grupo de arcos $L_{i j} = \{i j, j i\}$, en el que al menos uno de ellos debe ser atravesado en la solución final y además se busca un camino cuyo nodo inicial y final no se especifican.

\begin{figure}[tbp]
\centerline{\includegraphics[width=9.5cm]{expanded_graph.png}}
\caption{Expansión de un cruce entre una calle de sentido único y una calle de doble sentido. (a) Grafo mixto original $H$. (b) Grafo dirigido $G'$ luego de la expansión, los arcos auxiliares están representados por líneas discontinuas. [Fuente: \citet{Braier2017AnArgentina}]}
\label{fig:grafo_expandido}
\end{figure}

\subsection{Variables de decisión}
\begin{itemize}
\item $x_{i j}$: Para cada arco $ {i j} \in A$ esta variable representa el número de veces que $i j$ es atravesado.

\item $s_i$: Para cada nodo $i \in I$ esta variable binaria especifica si  $i$ es el primer nodo de la ruta.

\item $t_j$: Para cada nodo $j \in V$ esta variable binaria indica si $j$ es el último nodo de la ruta.
\end{itemize}

\subsection{Definición del programa entero}
\label{sec:programa-entero}
\begin{equation*} \tag{0} \label{eq0}
\min \sum_{i j \in A} w_{i j} x_{i j}  \\
\end{equation*} 
\hbox{}

\begin{equation} \tag{1} \label{eq1}
\begin{gathered}
x_{i j} \geq 1 \\
\forall i j \in A M
\end{gathered}
\end{equation} 
\hbox{}

\begin{equation} \tag{2} \label{eq2}
\begin{gathered}
x_{i j} + x_{j i} \geq 1 \\
\forall i j \in E
\end{gathered}
\end{equation}
\hbox{}

%ecuacion 3a
\begin{equation} \tag{3a} \label{eq3a}
\begin{gathered}
s_i + \sum_{j: j i \in A} x_{j i} = \sum_{j: i j \in A} x_{i j} + t_i \\
\forall i \in I
\end{gathered}
\end{equation} 
\hbox{}

%ecuacion 3b
\begin{equation} \tag{3b} \label{eq3b}
\begin{gathered}
\sum_{j: j i \in A} x_{j i} = \sum_{j: i j \in A} x_{i j} + t_i \\
\forall i \in V\backslash I
\end{gathered}
\end{equation}
\hbox{}

\begin{equation} \tag{4} \label{eq4}
\sum_{i \in I} s_i = 1 
\end{equation}
\hbox{}

\begin{equation} \tag{5} \label{eq5}
\sum_{i \in V} t_i = 1 
\end{equation}
\hbox{}

\begin{equation} \tag{6} \label{eq6}
\begin{gathered}
    \sum_{i j \in \delta + (\mathcal{S})} x_{i j} \geq 1 \\
    \forall \mathcal{S} \subseteq V
\end{gathered}
\end{equation}
\hbox{}

\begin{equation} \tag{7} \label{eq7}
\begin{gathered}
    x_{i j} \in \mathbb{Z}_+ \\
    \forall i j \in A
\end{gathered}
\end{equation}
\hbox{}

\begin{equation} \tag{8} \label{eq8}
\begin{gathered}
    s_i \in \{0,1\} \\
    \forall i \in I
\end{gathered}
\end{equation}
\hbox{}

\begin{equation} \tag{9} \label{eq9}
\begin{gathered}
    t_i \in \{0,1\} \\
    \forall i \in V
\end{gathered}
\end{equation}

La función objetivo (\ref{eq0}) busca minimizar el costo total de la ruta de recolección de residuos. El costo en este trabajo se refiere a la distancia recorrida por el vehículo recolector. La restricción (\ref{eq1}) impone que todos los arcos de único sentido deben ser visitados al menos una vez, la (\ref{eq2}) requiere que los arcos no dirigidos sean atravesados al menos una vez en cualquier sentido. Las restricciones (\ref{eq3a}) y (\ref{eq3b}) aseguran que la solución encontrada es realmente un camino agregando la condición de conservación de flujo estándar a cada nodo. Las restricciones (\ref{eq4}) y (\ref{eq5}) garantizan que el nodo inicial y final sean únicos. Las restricciones (\ref{eq1})-(\ref{eq5}) permiten la formación de subtours, la restricción (\ref{eq6}) es el estándar de eliminación de subtours. Las restricciones (\ref{eq7})-(\ref{eq9}) especifican los valores posibles para las variables del modelo.

\subsection{Algoritmo de solución}
\label{algoritmo-solucion}
% En el modelo dado en la sección \ref{sec:programa-entero} se puede observar la restricción de eliminación de subtours la cual es muy costosa en la generación de las mismas como también en  la complejidad del problema. En este contexto la estrategia propuesta en \cite{Braier2017AnArgentina} se basa en tratar de obtener una solución rápidamente sin considerar el problema de subtour e ir agregando las restricciones sucesivamente, esto se conoce como técnica de agregación dinámica de restricciones a un modelo relajado.

En el modelo dado en la sección \ref{sec:programa-entero} se puede observar la restricción de eliminación de subtours la cual es muy costosa en la generación de las mismas como también en la complejidad del problema. En este contexto la estrategia propuesta en \citet{Braier2017AnArgentina} se basa en tratar de obtener una solución rápidamente sin considerar el problema de subtour, en caso de que existan subtours se eliminan mediante el procedimientos de mezcla de subtours, si no se puede utilizar esta técnica se van agregando las restricciones sucesivamente, esto se conoce como técnica de agregación dinámica de restricciones a un modelo relajado.

Los siguientes pasos detallan la estrategia:

\begin{enumerate}
\item Crear el modelo relajado $R: = (\ref{eq1}) - (\ref{eq5})$ y $(\ref{eq7}) - (\ref{eq9})$.
\item Resolver $R$.
\item Si no se puede encontrar ninguna solución para $R$, retornar ``infactible'' y parar.
\item Si la mejor solución encontrada para $R$ no tiene subtours, retornar esta solución y parar.
\item Si los subtours pueden ser mezclados con el tour que contiene el nodo inicial y final, entonces mezclarlos, retornar la solución obtenida y parar.
\item De lo contrario, agregar a $R$ la restricción de eliminación de subtour estándar (\ref{eq6}) por cada subtour en la solución y regresar al Paso 2.
\end{enumerate}

El procedimiento de mezcla de subtours descrito por \citet{Braier2017AnArgentina}, utilizado en el paso 5, consiste en que dado un subtour \textit{T} y el camino principal \textit{P}, es decir el camino que empieza en el único vértice $i \in I$ con $s_i = 1$, y termina respectivamente en $t_i = 1$, se intenta intercambiar los arcos auxiliares para mezclar \textit{T} y \textit{P}. Se puede presentar una de las siguientes tres configuraciones:

\begin{itemize}
    \item Configuración A: Si el camino principal y el subtour se encuentran en algún nodo intermedio, como en la Figura \ref{fig:procedimiento_mezcla_subtours}(a), entonces son unidos como en la Figura \ref{fig:procedimiento_mezcla_subtours}(b)
    \item Configuración B: Si el subtour se encuentra con el último nodo en la ruta principal, como en la Figura \ref{fig:procedimiento_mezcla_subtours}(c), entonces se unen como en la Figura \ref{fig:procedimiento_mezcla_subtours}(d).
    \item Configuración C: Si el subtour se encuentra con el primer nodo en la ruta principal, como en la Figura \ref{fig:procedimiento_mezcla_subtours}(e), entonces se unen como en la Figura \ref{fig:procedimiento_mezcla_subtours}(f).
\end{itemize}

\begin{figure}[tbp]
\centerline{\includegraphics[width=\textwidth]{mezcla_subtours.png}}
\caption{Posibles configuraciones en el procedimiento de la mezcla de subtours. Si \textit{P} y \textit{T} tienen la configuración (a), (b) o (c), entonces la solución es modificada de acuerdo a lo especificado en (b), (d) o (f) respectivamente. [Fuente: \citet{Braier2017AnArgentina}]}
\label{fig:procedimiento_mezcla_subtours}
\end{figure}

Este procedimiento es aplicado para cada subtour en la solución hasta que todos sean mezclados al camino principal, dado que el costo de todos los arcos auxiliares que unen los nodos de las esquinas es el mismo, la ruta modificada permanece óptima, por lo que el algoritmo se detiene y devuelve la solución obtenida en este caso.

Dado que no siempre los subtours pueden ser mezclados con la técnica recién mencionada, el algoritmo principal descrito agrega una restricción estándar de eliminación de subtoures al modelo $R$, tal y como se especifica en el Paso 6.

Finalmente, en caso que el modelo matemático encuentre una solución factible, el resultado del modelo indica la distancia total, los arcos que son atravesados y el número de veces que estos son atravesados. Sin embargo, para conocer el camino a seguir es necesario encontrar la secuencia del mismo. 

\section{Método de secuenciación}

% En este trabajo se realiza una búsqueda en profundidad para obtener la secuencia a seguir a partir del resultado del problema descrito en la sección anterior. El algoritmo toma como entrada el nodo inicial, nodo final y la cantidad de veces que se atraviesa por cada arco del grafo $G$. Se describe a continuación los pasos del algoritmo:

% En este trabajo se realiza una búsqueda en profundidad para obtener la secuencia a seguir a partir del resultado del problema descrito en la sección anterior. El algoritmo recibe como parámetros de entrada el nodo inicial, nodo final, la cantidad de veces que se atraviesa por cada arco del grafo $G$ y un grafo dirigido $N$ construido teniendo en cuenta solo los arcos atravesados. Se describe a continuación los pasos del algoritmo:

% \begin{enumerate}
% \item Leer el grafo $N$ resultante del GDRPP abierto. 
% \item Inicializar nodo actual al valor del nodo inicial.
% \item Inicializar un vector vacío $seq$, en el que los nodos se almacenarán en el orden en que deben ser visitados.
% \item Agregar a $seq$ el nodo actual.
% \item Si el nodo actual es igual al nodo final y ya se atravesaron todos los arcos, entonces retornar $seq$ y parar.
% \item Sino, tomar uno de los nodos sucesores del nodo actual en $N$, donde el arco formado por el nodo actual y el nodo sucesor aún no haya sido tenido en cuenta, asignar como nodo actual el nodo sucesor y volver al paso 4.
% \end{enumerate}

En este trabajo se utiliza el algoritmo implementado en \cite{RiveraHazim2015APath} que encuentra el camino euleriano de un multigrafo dirigido para obtener la secuencia a seguir. Se genera un multigrafo dirigido $MG$ a partir del resultado del problema descrito en la sección anterior, creando arcos dirigidos según la cantidad de veces que se atraviesa por cada arco del grafo $G'$ . 

El algoritmo recibe como parámetros de entrada el nodo inicial, nodo final y $MG$. Devuelve como resultado, la secuencia de arcos del recorrido solución. Los siguientes pasos detallan el algoritmo de secuenciación:

\begin{enumerate}
    \item Comience con una pila vacía y un camino (euleriano) vacío. 
    \item Se haya el multigrafo reverso de $MG$ y se trabajo con el mismo. El reverso es un grafo con los mismos nodos y bordes pero con las direcciones de los arcos invertidas.
    \item Se elige el vértice que representa el nodo final.
    \item Si el vértice actual no tiene arcos salientes (es decir, vecinos): agréguelo al camino, elimine el último vértice de la pila y configúrelo como el actual. De lo contrario (en caso de que tenga arcos salientes, es decir, vecinos): agregue el vértice a la pila, tome cualquiera de sus vecinos, elimine el arco entre ese vértice y el vecino seleccionado, y establezca ese vecino como el vértice actual.
    \item Repita el Paso 4 hasta que el vértice actual no tenga más arcos salientes (vecinos) y la pila esté vacía.
\end{enumerate}

La complejidad del algoritmo es $O(N + M)$, donde $N$ es el número de vértices y $M$ es el número de arcos.

\section{Ejemplo Numérico}
A continuación se describe un ejemplo simple de cómo la aplicación produce la secuencia del camino a seguir desde el resultado generado por la programación matemática.

El Paso 1 de la Figura \ref{fig:PasosSolucion} muestra los datos de salida que genera el modelo matemático (Algoritmo  de  solución, sección \ref{algoritmo-solucion}) a partir de un resultado factible. La salida contiene todos los segmentos que son utilizados como solución junto con el número de veces que se recorre cada uno de ellos, así como también despliega los nodos que resultaron elegidos como nodo inicial y nodo final del recorrido. 

Posteriormente, con estos datos se procede a crear el multigrafo dirigido $MG$ con la cantidad total de arcos resultantes (Paso 2). El multigrafo generado pasa a formar parte de la entrada, junto con los nodos inicial y final, para obtener la secuencia del camino a seguir (Paso 3). Por último, el algoritmo de secuenciación se resuelve y genera como salida la secuencia de arcos del algoritmo de optimización.

\begin{figure}[tb]
\centerline{\includegraphics[width=\textwidth]{pasos_de_solucion.png}}
\caption{Pasos de solución para obtener la secuencia del camino del vehículo recolector en una zona}
\label{fig:PasosSolucion}
\end{figure}


 % Propuesta

\clearpage
\lhead{\emph{Resultados y Discusión}} 
\renewcommand\chaptername{Capítulo}%título "Capítulo"
\chapter{Resultados y Discusión}
\label{resultadosdiscusion}
\ifpdf
  \graphicspath{{Chapter6/Chapter6Figs/PNG/}{Chapter6/Chapter6Figs/PDF/}{Chapter6/Chapter6Figs/}}
\else
  \graphicspath{{Chapter6/Chapter6Figs/EPS/}{Chapter6/Chapter6Figs/}}
\fi

\markboth{\hfill \thechapter. Resultados y Discusión}{\hfill \thechapter. Resultados y Discusión}

\section{Ambiente de Ejecución.}

Para la ejecución del proceso de optimización se utilizaron equipos con la siguiente configuración:
\begin{itemize}
\item Una computadora de escritorio con procesador Intel Core i5 de cuatro núcleos, con 8 GB de memoria RAM, y sistema operativo Windows 7 de 64 bits.
\item Una computadora de escritorio con procesador Intel Core i5 de cuatro núcleos, con 4 GB de memoria RAM, y sistema operativo Windows 7 de 64 bits.
\end{itemize}

Para la implementación del \textit{SMPSO-CLAHE}, se utilizaron librerías libres disponibles en la web. La implementación de la metaheurística \textit{PSO multi-objetivo} se encuentra escrita en el lenguaje JAVA, es una adaptación de la implementación original planteada y disponible en las librerías $jMetal$ \cite{Durillo2011}. Fue modificada para computar el cálculo del $fitness$, basándonos en que se desea maximizar la cantidad de información de la imagen y minimizar la distorsión de la misma.

Para la implementación del algoritmo \textit{CLAHE} y las métricas de evaluación, Entropía (\ref{sec:entropia}), Entropía Local (\ref{sec:entropialocal}), \textit{SSIM} (\ref{sec:ssim}) y \textit{LTG} (\ref{sec:ltg}), se toman como base las implementaciones existentes en \textit{Matlab} \cite{MatlabOTB}. 

Para la interoperabilidad entre las implementaciones del \textit{SMPSO multi-objetivo}, $CLAHE$ y las métricas de evaluación, se utilizó un esquema de intercambio de mensajes via socket (TCP/IP Socket Communications). 

Las pruebas se realizaron empleando 30 imágenes radiológicas previamente digitalizadas del tórax obtenidas del sitio https://openi.nlm.nih.gov/. Las mismas se seleccionaron a partir de la cantidad de detalles que poseen, lo que representa un desafío adecuado para la {\it mejora del contraste}.

Se realizaron 30 ejecuciones de \textit{SMPSO-CLAHE} por cada imagen de prueba, con un tamaño de enjambre de 70 partículas y el tamaño del archivo de líderes fue de 70. Se obtuvieron aproximadamente 300 imágenes soluciones Pareto por cada una de ellas, las cuales fueron nuevamente filtradas una vez terminadas las ejecuciones.


\section{Resultados Obtenidos.}

Los resultados experimentales obtenidos de las correlaciones entre los pares de métricas utilizados se muestran en las \textbf{Tablas \ref{tabla:correlacionFrontal} y \ref{tabla:correlacionLateral}} , donde los valores marcados en negrita demuestran la fuerte relación inversa lineal existente entre los pares de métricas, lo cual indica que estas métricas se complementan para mantener el compromiso entre aumento de contraste y minimización de la distorsión. El {\it coeficiente de correlación de Pearson}, obtenido a través de los coeficientes de las funciones objetivo durante las pruebas, muestra el comportamiento en términos de cuánto una métrica afecta a la otra debido al proceso de {\it mejora del contraste}.

\begin{table}[H]
\begin{center}
\caption{Resultados de promediar la correlación de Pearson usando Entropía ($\mathscr{H}$), Entropía Local ($\mathscr E$), \textit{SSIM} y \textit{LTG}, para imágenes de tórax frontal.}
 \begin{tabular}{|c|c|c|c|c|}
            \hline
            Métricas & $\mathscr{H}$ & $\mathscr{E}$ & \textit{SSIM}\\
            \hline
            $\mathscr{E}$ & -0.83918  &  &  \\ \hline
            \textit{SSIM} & -0.84739 & \textbf{-0.97740} & \\ \hline
            \textit{LTG} & -0.81365 & -0.96307 &  0.00923   \\ \hline
            \end{tabular}
\label{tabla:correlacionFrontal}
\end{center}
\end{table}

\begin{table}[H]
\begin{center}
\caption{Resultados de promediar la correlación de Pearson usando Entropía ($\mathscr{H}$), Entropía Local ($\mathscr E$), \textit{SSIM} y \textit{LTG}, para imágenes de tórax lateral.}
\begin{tabular}{|c|c|c|c|c|}
            \hline
            Métricas & $\mathscr{H}$ & $\mathscr{E}$ & \textit{SSIM} \\
            \hline
            $\mathscr{E}$ & -0.24288  &  &   \\ \hline
            \textit{SSIM} & -0.79984 & \textbf{-0.95521} &  \\ \hline
            \textit{LTG} & -0.39579 & -0.84976 & 0.33629    \\ \hline
            \end{tabular}
\label{tabla:correlacionLateral}
\end{center}
\end{table}

De acuerdo a los resultados obtenidos en la correlación de Pearson, se determina la fuerza de las relaciones entre las métricas \textit{Entropía Local} y \textit{SSIM}, ambas métricas demuestran ser las más contradictorias. Esto no significa que las demás métricas no sean contradictorias, simplemente optamos por el mayor valor de contradicción, según la correlación obtenida, que son el par de métricas \textit{Entropía Local} ($\mathscr{E}$) e \textit{Índice de Similitud Estructural} (\textit{SSIM}), esto indica que el mejoramiento de una función objetivo es logrado a costa del empeoramiento de la otra función objetivo en un contexto de minimización o maximización de ambas.

Se utilizó la correlación de Pearson para medir el grado de relación  de los pares de métricas, que fueron utilizados como funciones objetivos en el proceso de optimización Robusta.


En la \textbf{Figura  \ref{fig:resultado_entropia_local_ssim_img1}} se muestran las soluciones obtenidas correspondientes a una imagen de tórax frontal y en la \textbf{Figura  \ref{fig:resultado_entropia_local_ssim_img9}} se muestran las soluciones obtenidas correspondientes a una imagen de tórax lateral, que se encuentran en el Conjunto Pareto para las métricas {\it Entropía Local/SSIM}, además de la imágenes originales como referencia visual.

%Imagen 1 torax frontal entropia local/ssim
\begin{figure}[H]
    \begin{center}
        \subfigure[][\label{fig:label:a} $SSIM=1$ \space $\mathscr{E}=3.4754$
        ]{\includegraphics[width=4.5cm]{entropia_local_ssim/imagen1.png}}
        \subfigure[][\label{fig:label:b} $SSIM=0.7402$ \space $\mathscr{E}=4.5639$]{\includegraphics[width=4.5cm]{entropia_local_ssim/imagen1_193_41_0_local_ssim.png}}
        \subfigure[][\label{fig:label:c} $SSIM=0.9724$ \space  $\mathscr{E}=3.8264$]{\includegraphics[width=4.5cm]{entropia_local_ssim/imagen1_210_3_0_local_ssim.png}}
    \end{center}
    \caption{Tórax frontal 1.}
    \captionsetup{aboveskip=0pt}
    \caption*{ Imagen Original \ref{fig:label:a}, Imágenes resultantes \ref{fig:label:b} y \ref{fig:label:c}}
    \label{fig:resultado_entropia_local_ssim_img1}
\end{figure}

%Imagen 9 torax lateral entropia local/ssim
\begin{figure}[hbtp]
    \begin{center}
        \subfigure[][\label{fig:label:d} $SSIM = 1$ \space $\mathscr{E}=2.8157$
        ]{\includegraphics[width=4.5cm]{entropia_local_ssim/imagen9.png}}
        \subfigure[][\label{fig:label:e} $SSIM=0.8871$ $\mathscr{E}=3.4791$]{\includegraphics[width=4.5cm]{entropia_local_ssim/imagen9_2_256_0-005186957854_local_ssim.png}}
        \subfigure[][\label{fig:label:f} $SSIM=0.9016$ $\mathscr{E}=3.4296$]{\includegraphics[width=4.5cm]{entropia_local_ssim/imagen9_49_82_0_local_ssim.png}}
    \end{center}
    \caption{Tórax lateral 2.}
    \captionsetup{aboveskip=0pt}
    \caption*{ Imagen Original \ref{fig:label:d}, Imágenes resultantes \ref{fig:label:e} y \ref{fig:label:f}}
    \label{fig:resultado_entropia_local_ssim_img9}
\end{figure}

La relación inversa entre las métricas se refleja en los resultados obtenidos. A partir de las Figuras \ref{fig:label:c} y \ref{fig:label:f} se observa que a medida que la métrica {\it SSIM} se aproxima a \textbf{1} los resultados se asemejan más a las imágenes originales, Figuras \ref{fig:label:a} y \ref{fig:label:d}, en términos de contraste, y de visibilidad de detalles; en cambio, mientras la {\it Entropía Local} aumenta se diferencian más los detalles no visibles debido al bajo contraste, Figuras \ref{fig:label:b} y \ref{fig:label:e}. 

Las imágenes que conforman el Conjunto Pareto resaltan distintos detalles a medida que los objetivos varían, lo cual se logra a partir de que se asegura la selección de las métricas más adecuadas para la optimización, basada en el análisis descrito anteriormente.

Las imágenes se dividieron en dos grupos, el primer grupo, imágenes de tórax lateral y el segundo grupo, imágenes de tórax frontal.

Se analizó el comportamiento del Conjunto Pareto resultante de cada imagen procesada, se consideró el {\it Frente Pareto óptimo} de una imagen $f$, como el conjunto de variables de decisión de las {\it N-1} imágenes y se obtuvo un nuevo {\it Frente Pareto} para las {\it N-1} imágenes. También se consideró como una sola entrada el conjunto de todas las imágenes y se realizó el cálculo de la Entropía Local \ref{sec:entropialocal} y SSIM \ref{sec:ssim} entre todas las imágenes de entrada; obteniendo así el {\it Frente Pareto Robusto}. De esta forma podemos observar si los valores del {\it Frente Pareto} de la imagen $f$ varían de forma significativa usando como entrada otras imágenes o son óptimas para las imágenes de referencia del mismo grupo.

% En la \textbf{Figura \ref{fp-original-resultante-tf}} se muestran el {\it Frente Pareto} óptimo de la imagen \textbf{\ref{fig:label:a}} y los Frentes Pareto sub-óptimos al variar las variables de decisión, tomando los Frentes Paretos de las otras imágenes del grupo. Se puede observar que algunos Frentes Pareto se desplazan con respecto al original.

En la \textbf{Figura \ref{fp-original-resultante-tf}} se muestran el {\it Frente Pareto} óptimo de la imagen \textbf{\ref{fig:label:a}} y los Frentes Pareto sub-óptimos al variar las variables de decisión de las imágenes del mismo grupo, tomando como soluciones no dominadas el Frente Pareto de la imagen \textbf{\ref{fig:label:a}}. Se puede observar que algunos Frentes Pareto se desplazan con respecto al Frente Pareto Óptimo.

%figuras FP entropia local/ssim tórax frontal imagen 1
\begin{figure}[H]
  \begin{center}
    \leavevmode
    \includegraphics[width=14cm]{Chapter6/Chapter6Figs/FP-torax_frontal/img1-FP-variado-optimo.png}
    \caption {Resultados sub-óptimos obtenidos al aplicar los parámetros de optimización de la imagen \ref{fig:label:a} a las otras imágenes del mismo grupo (Tórax Frontal) \\
    {{\color{yellow} {$\bullet$}} Frente pareto óptimo}}
    \label{fp-original-resultante-tf}
  \end{center}
\end{figure}

En la \textbf{Figura \ref{fig:FP-robusto-sub-optimos}}, se comparan el Frente Pareto Robusto del grupo de imágenes de Tórax frontal y los Frentes Pareto sub-óptimos obtenidos por el proceso de prueba sobre todas las imágenes del grupo Tórax Frontal, utilizando las soluciones no dominadas de la imagen \textbf{\ref{fig:label:a}}, se puede observar que los resultados no son óptimos, pero son lo suficientemente válidos para el grupo de imágenes estudiados.

% Los parámetros que se obtuvieron en la Optimización Robusta son aplicables a cualquier imagen del grupo estudiado, a diferencia de trabajos anteriores, donde los parámetros se utilizaban para una sola imagen.

Además, se verifica que el Frente Pareto robusto cae en su óptimo con respecto a la optimización realizada sobre una sola imagen, pero en general se comporta de manera satisfactoria para el grupo de imágenes analizadas. Con esto también se observa que la solución óptima de una imagen $f$ no necesariamente es la mejor solución para una imagen $f_n$.

% %figuras FP entropia local/ssim tórax frontal imagen 2
\begin{figure}[H]
  \begin{center}
    \leavevmode
    \includegraphics[width=14cm]{Chapter6/Chapter6Figs/FP-torax_frontal/img1-FP-variado-Frontal-robusto.png}
    \caption {Frente Pareto Entropía Local - SSIM. Tórax frontal\\}{
    Resultados sub-óptimos obtenidos al aplicar los parámetros de optimización de la imagen \ref{fig:label:a} a las otras imágenes del mismo grupo (Tórax Frontal).\\
    {{\color{myorange} {$\bullet$}} Frente pareto robusto}}
    \label{fig:FP-robusto-sub-optimos}
  \end{center}
\end{figure}

\begin{figure}[H]
  \begin{center}
    \leavevmode
    \includegraphics[width=14cm]{Chapter6/Chapter6Figs/FP-torax_frontal/img1-FP-variado-optimo-robusto.png}
    \caption {Frente Pareto Entropía Local - SSIM. Tórax frontal\\}{
    Resultados sub-óptimos obtenidos al aplicar los parámetros de optimización de la imagen \ref{fig:label:a} a las otras imágenes del mismo grupo (Tórax Frontal).\\
    {{\color{myorange} {$\bullet$}} Frente pareto robusto} \\
    {{\color{yellow} {$\bullet$}} Frente pareto óptimo}}
    \label{fig:FP-robusto-sub-optimos-optimo}
  \end{center}
\end{figure}


Para evaluar los resultados del Conjunto Pareto se utilizó la medición del hipervolumen \cite{hypervolume}. Esta métrica mide el área (2 objetivos) del espacio objetivo que es cubierto por el conjunto de soluciones, que está limitado por un punto de referencia. Esto equivale a la suma de todas las areas rectangulares, delimitadas por un punto de referencia. {\it yref}.

Cuando más grande es el valor del hipervolumen, se dice que es más dominante el conjunto de puntos del Frente Pareto. Esto hace que el hipervolumen sea una medida razonable para medir que tan eficiente es un Frente Pareto \cite{shah2016pareto}.

Para el punto de referencia se optó por el punto [6,1].

En la \textbf{Tablas \ref{tabla:hipervolumen-robustos} y \ref{tabla:hipervolumen-promedio-optimo}} se puede ver que el Hipervolumen cubierto por el Frente Pareto Robusto es similar al Hipervolumen cubierto por los Frentes Paretos Óptimos de las imágenes; es decir, el Frente Pareto Robusto no se encuentra muy lejos de los Frentes Paretos Óptimos. 
\begin{table}[H]
\centering
\caption{Hipervolumen de los Frentes Paretos Robustos.}
\begin{tabular}{|c|c|}
\hline
Frente Pareto Robusto & Valor Hipervolumen {\it yref}: [6,1] \\
\hline \hline 
\it{Tórax Frontal} & 0.99871 \\ \hline
\it{Tórax Lateral}  & 1.51317\\ \hline
\end{tabular}
\label{tabla:hipervolumen-robustos}
\end{table}

\begin{table}[H]
\centering
\caption{Hipervolumen promediado de los Frentes Paretos Óptimos.}
\begin{tabular}{|c|c|}
\hline
Frente Pareto Óptimo & Valor Hipervolumen {\it yref}: [6,1] \\
\hline \hline 
\it{Grupo Tórax Frontal} & 0,94578 \\ \hline
\it{Grupo Tórax Lateral}  & 1,45684\\ \hline
\end{tabular}
\label{tabla:hipervolumen-promedio-optimo}
\end{table} % Resultados y Discusión

\clearpage
\lhead{\emph{Conclusión}} 
\renewcommand\chaptername{Capítulo}%título "Capítulo"
\def\baselinestretch{1}
\chapter{Conclusiones y Trabajos Futuros}

\markboth{\hfill \thechapter. Conclusiones y Trabajos Futuros}{\hfill \thechapter. Conclusiones y Trabajos Futuros}

\section {Conclusiones del trabajo}

Se analizaron varias métricas para determinar la calidad de la imagen, basados en un enfoque de referencia completa \textbf{full reference} (FR), se seleccionaron 4 métricas, las cuales se trabajaron de a pares {\it Entropía/SSIM}, {\it Entropía local/SSIM}, {\it Entropía/LTG} y {\it Entropía local/LTG}, junto al algoritmo $SMPSO-CLAHE$, con el objetivo de obtener las métricas que maximicen el contraste y minimicen la distorsión de la imagen de manera simultánea.

A partir de las pruebas realizadas y de los resultados obtenidos, se pueden considerar las siguientes conclusiones:

\begin{itemize}
\item Los resultados experimentales obtenidos en la \textbf{Tabla \ref{tabla:promcorrelacion}} muestran que los pares de métricas {\it Entropía local/SSIM} demuestran ser los más contradictorios según la correlación obtenida, por tanto son más adecuados para incorporar a un proceso de optimización. \newline
\item Los resultados de las imágenes para SMPSO-CLAHE muestran una mejora en el contraste, manteniendo la apariencia natural de las mismas. Este algoritmo se muestra aplicable tanto en imágenes médicas o biométricas, mostrando resultados satisfactorios.\newline
\item Los parámetros que se obtuvieron en la Optimización Robusta son aplicables a cualquier imagen del grupo estudiado, a diferencia de trabajos anteriores, donde los parámetros se utilizaban para una sola imagen.
% \item Este trabajo es un caso general del trabajo de Moré y Brizuela \cite{morebrizuela2014}, cuyos resultados caen en el frente pareto de esta propuesta, y cuya solución corresponde a la mínima similaridad. No se consideraron pruebas experimentales debido a que son enfoques diferentes.
\end{itemize}


\section{Trabajos Futuros}
Como trabajos futuros de manera a seguir con esta tesis de grado se propone: 

\begin{itemize}
\item Utilizar la implementación {\it SMPSO-CLAHE} con otras métricas de evaluación de calidad y hallar el índice de correlación.
\item Utilizar otros índices de Correlación y realizar una comparación con la utilizada en este trabajo.
\item Analizar el desempeño de esta propuesta en imágenes de otra naturaleza.
\end{itemize}

 % Conclusión

%% --------------- Apéndices --------------------------------------

\clearpage
\appendix
\lhead{\emph{Apéndice A}} 
\renewcommand{\appendixname}{Apéndice}
\chapter\bigskip
\ifpdf
  \graphicspath{{AppendixA/AppendixAFigs/PNG/}{AppendixA/AppendixAFigs/PDF/}{AppendixA/AppendixAFigs/}}
\else
  \graphicspath{{AppendixA/AppendixAFigs/EPS/}{AppendixA/AppendixAFigs/}}
\fi

\section{Valores de las métricas seleccionadas en las imágenes de referencias.}
\begin{figure}[H]
    \begin{center}
        \subfigure[][\label{fig:img1} $SSIM = 1$ \newline $LTG = 1 $ \newline $\mathscr{H} = 7.4428$ \newline $\mathscr E =3.4754$]
        {\includegraphics[width=4.5cm]{AppendixA/AppendixAFigs/originales/frontales/imagen1.png}}
        \subfigure[][\label{fig:img2} $SSIM = 1$ \newline $LTG = 1$ \newline $\mathscr{H} = 6.6690$ \newline $\mathscr E =2.9755$
        ]{\includegraphics[width=4.5cm]{AppendixA/AppendixAFigs/originales/frontales/imagen2.png}}
        \subfigure[][\label{fig:img4} $SSIM = 1$ \newline $LTG = 1$ \newline $\mathscr{H} = 7.1196$ \newline $\mathscr E =3.4570$
        ]{\includegraphics[width=4.5cm]{AppendixA/AppendixAFigs/originales/frontales/imagen4.png}}
    \end{center}
    \label{fig:gral1}
\end{figure}

\begin{figure}[H]
    \begin{center}
        \subfigure[][\label{fig:img3} $SSIM = 1$ \newline $LTG = 1$ \newline $\mathscr{H} = 7.3003$ \newline $\mathscr E =3.7876$
        ]{\includegraphics[width=4.5cm]{AppendixA/AppendixAFigs/originales/frontales/imagen3.png}}
        \subfigure[][\label{fig:img5} $SSIM = 1$ \newline $LTG = 1$ \newline $\mathscr{H} = 7.4409$ \newline $\mathscr E =3.5873$
        ]{\includegraphics[width=4.5cm]{AppendixA/AppendixAFigs/originales/frontales/imagen5.png}}
        \subfigure[][\label{fig:img7} $SSIM = 1$ \newline $LTG = 1$ \newline $\mathscr{H} = 7.7184$ \newline $\mathscr E =3.9015$
        ]{\includegraphics[width=4.5cm]{AppendixA/AppendixAFigs/originales/frontales/imagen7.png}}
    \end{center}
    \label{fig:gral2}
\end{figure}

\begin{figure}[H]
    \begin{center}
        \subfigure[][\label{fig:img6}  $SSIM = 1$ \newline $LTG = 1$ \newline $\mathscr{H} = 6.8287$ \newline $\mathscr E =3.1996$
        ]{\includegraphics[width=4.5cm]{AppendixA/AppendixAFigs/originales/frontales/imagen6.png}}
        \subfigure[][\label{fig:img8}  $SSIM = 1$ \newline $LTG = 1$ \newline $\mathscr{H} = 6.8998$ \newline $\mathscr E =3.3196$
        ]{\includegraphics[width=3.0cm]{AppendixA/AppendixAFigs/originales/frontales/imagen8.png}}
        \subfigure[][\label{fig:img9}  $SSIM = 1$ \newline $LTG =1 $ \newline $\mathscr{H} = 6.6672$ \newline $\mathscr E =3.4084$
        ]{\includegraphics[width=4.5cm]{AppendixA/AppendixAFigs/originales/frontales/imagen9.png}}
    \end{center}
    \label{fig:gral3}
\end{figure}

\begin{figure}[H]
    \begin{center}
        \subfigure[][\label{fig:img10}  $SSIM = 1$ \newline $LTG = 1$ \newline $\mathscr{H} = 6.8803$ \newline $\mathscr E =3.3658$
        ]{\includegraphics[width=4.5cm]{AppendixA/AppendixAFigs/originales/frontales/imagen10.png}}
        \subfigure[][\label{fig:img11}  $SSIM = 1$ \newline $LTG = 1$ \newline $\mathscr{H} = 7.5582$ \newline $\mathscr E =3.5560$
        ]{\includegraphics[width=3.7cm]{AppendixA/AppendixAFigs/originales/frontales/imagen11.png}}
        \subfigure[][\label{fig:img12}  $SSIM = 1$ \newline $LTG = 1$ \newline $\mathscr{H} = 6.5775$ \newline $\mathscr E =3.02324.1202$
        ]{\includegraphics[width=4.5cm]{AppendixA/AppendixAFigs/originales/frontales/imagen12.png}}
    \end{center}
    \label{fig:gral4}
\end{figure}

\begin{figure}[H]
    \begin{center}
        \subfigure[][\label{fig:img13}  $SSIM = 1$ \newline $LTG = 1$ \newline $\mathscr{H} = 7.5991$ \newline $\mathscr E =4.1202$
        ]{\includegraphics[width=4.5cm]{AppendixA/AppendixAFigs/originales/frontales/imagen13.png}}
        \subfigure[][\label{fig:img14}  $SSIM = 1$ \newline $LTG = 1$ \newline $\mathscr{H} = 6.9048$ \newline $\mathscr E =3.4270$
        ]{\includegraphics[width=4.5cm]{AppendixA/AppendixAFigs/originales/frontales/imagen14.png}}
        \subfigure[][\label{fig:img15}  $SSIM = 1$ \newline $LTG = 1$ \newline $\mathscr{H} = 6.7262$ \newline $\mathscr E =3.1108$
        ]{\includegraphics[width=4.5cm]{AppendixA/AppendixAFigs/originales/frontales/imagen15.png}}
    \end{center}
    \label{fig:gral5}
\end{figure}


\begin{figure}[H]
    \begin{center}
        \subfigure[][\label{fig:imgtll} $SSIM = 1$ \newline $LTG = 1 $ \newline $\mathscr{H} = 7.2447$ \newline $\mathscr E = 2.8626$]
        {\includegraphics[width=4cm]{AppendixA/AppendixAFigs/originales/laterales/imagen1.png}}
        \subfigure[][\label{fig:imgtl2} $SSIM = 1$ \newline $LTG = 1$ \newline $\mathscr{H} =7.1010 $ \newline $\mathscr E = 2.8157$
        ]{\includegraphics[width=4cm]{AppendixA/AppendixAFigs/originales/laterales/imagen2.png}}
        \subfigure[][\label{fig:imgtl4} $SSIM = 1$ \newline $LTG = 1 $ \newline $\mathscr{H} = 6.7004$ \newline $\mathscr E = 2.7408$]
        {\includegraphics[width=4cm]{AppendixA/AppendixAFigs/originales/laterales/imagen4.png}}
    \end{center}
    \label{fig:gral6}
\end{figure}

\begin{figure}[H]
    \begin{center}
       
        \subfigure[][\label{fig:imgtl6} $SSIM = 1$ \newline $LTG = 1$ \newline $\mathscr{H} = 5.8141$ \newline $\mathscr E = 1.9030$
        ]{\includegraphics[width=4cm]{AppendixA/AppendixAFigs/originales/laterales/imagen6.png}}
       \subfigure[][\label{fig:imgtl7} $SSIM = 1$ \newline $LTG = 1 $ \newline $\mathscr{H} = 6.9070$ \newline $\mathscr E = 2.9594$]
        {\includegraphics[width=4cm]{AppendixA/AppendixAFigs/originales/laterales/imagen7.png}}
        \subfigure[][\label{fig:imgtl8} $SSIM = 1$ \newline $LTG = 1$ \newline $\mathscr{H} = 7.0622$ \newline $\mathscr E = 2.9446$
        ]{\includegraphics[width=4cm]{AppendixA/AppendixAFigs/originales/laterales/imagen8.png}}
    \end{center}
    \label{fig:gral7}
\end{figure}

\begin{figure}[H]
    \begin{center}
        
        \subfigure[][\label{fig:imgtl9} $SSIM = 1$ \newline $LTG = 1$ \newline $\mathscr{H} = 7.5625$ \newline $\mathscr E = 3.4395$
        ]{\includegraphics[width=4cm]{AppendixA/AppendixAFigs/originales/laterales/imagen9.png}}
        \subfigure[][\label{fig:imgtl10} $SSIM = 1$ \newline $LTG = 1 $ \newline $\mathscr{H} = 7.1064$ \newline $\mathscr E = 3.3058$]
        {\includegraphics[width=4cm]{AppendixA/AppendixAFigs/originales/laterales/imagen10.png}}
        \subfigure[][\label{fig:imgtl11} $SSIM = 1$ \newline $LTG = 1$ \newline $\mathscr{H} = 6.2748$ \newline $\mathscr E = 2.4087$
        ]{\includegraphics[width=4cm]{AppendixA/AppendixAFigs/originales/laterales/imagen11.png}}
    \end{center}
    \label{fig:gral8}
\end{figure}

\begin{figure}[H]
    \begin{center}
        \subfigure[][\label{fig:imgtl3} $SSIM = 1$ \newline $LTG = 1$ \newline $\mathscr{H} = 5.7765$ \newline $\mathscr E = 2.4201$
        ]{\includegraphics[width=4.5cm]{AppendixA/AppendixAFigs/originales/laterales/imagen3.png}}
        \subfigure[][\label{fig:imgtl5} $SSIM = 1$ \newline $LTG = 1$ \newline $\mathscr{H} = 6.4335$ \newline $\mathscr E =2.7586$
        ]{\includegraphics[width=4.5cm]{AppendixA/AppendixAFigs/originales/laterales/imagen5.png}}
    \end{center}
    \label{fig:gral9}
\end{figure}


\section{Resultado Correlación de Pearson}

\begin{table}[H]
\centering
\caption{Promedio de la correlación de Pearson.}
\begin{tabular}{|c|c|}
\hline
Métricas & Correlación \\
\hline \hline 
%-0.9547657122	-0.9866466591	-0.9308066035	-0.9731620476
$\mathscr{H}$ / \textit{SSIM} & -0.9554 \\ \hline
$\mathscr{E}$ / \textit{SSIM} & \textbf{-0.9870}\\ \hline
$\mathscr{H}$ / \textit{LTG} & -0.9319 \\ \hline
$\mathscr{E}$ / \textit{LTG} & -0.9731 \\ \hline
\end{tabular}
\label{tabla:promcorrelacion}
\centering
\end{table}

\section{Resultados Indicador de Hipervolumen}

\begin{table}[H]
\centering
\caption{Hipervolumen de los Frentes Pareto - Tórax Frontal.}
\begin{tabular}{|c|c|}
\hline
Frente Pareto & Valor Hipervolumen {\it yref}: [6,1] \\
\hline \hline 
\it{Robusto} &  0.9987\\ \hline
\it{Tórax Frontal \ref{fig:img1}} & 0.97088 \\ \hline
\it{Tórax Frontal \ref{fig:img2}} & 1.07444 \\ \hline
\it{Tórax Frontal \ref{fig:img4}} & 0.94675 \\ \hline
\it{Tórax Frontal \ref{fig:img3}} & 0.98474 \\ \hline
\it{Tórax Frontal \ref{fig:img5}} & 1.18447 \\ \hline
\it{Tórax Frontal \ref{fig:img7}} & 0.89804 \\ \hline
\it{Tórax Frontal \ref{fig:img6}} & 1.05435 \\ \hline
\it{Tórax Frontal \ref{fig:img8}} & 0.94021 \\ \hline
\it{Tórax Frontal \ref{fig:img9}} & 0.79659 \\ \hline
\it{Tórax Frontal \ref{fig:img10}} & 0.89486 \\ \hline
\it{Tórax Frontal \ref{fig:img11}} & 0.97175 \\ \hline
\it{Tórax Frontal \ref{fig:img12}} & 1.08461 \\ \hline
\it{Tórax Frontal \ref{fig:img13}} & 0.62923 \\ \hline
\it{Tórax Frontal \ref{fig:img14}} & 0.85066 \\ \hline
\it{Tórax Frontal \ref{fig:img15}} & 0.90512 \\ \hline
\end{tabular}
\label{tabla:hipervolumen-frontal}
\end{table}

\begin{table}[H]
\centering
\caption{Hipervolumen de los Frentes Pareto - Tórax Lateral.}
\begin{tabular}{|c|c|}
\hline
Frente Pareto & Valor  Hipervolumen {\it yref}: [6,1] \\
\hline \hline 
\it{Robusto} &  1.51317\\ \hline
\it{Tórax Lateral \ref{fig:imgtll}} & 1.29874 \\ \hline
\it{Tórax Lateral \ref{fig:imgtl2}} & 1.47126 \\ \hline
\it{Tórax Lateral \ref{fig:imgtl4}} & 1.34961 \\ \hline
\it{Tórax Lateral \ref{fig:imgtl6}} & 1.77041 \\ \hline
\it{Tórax Lateral \ref{fig:imgtl7}} & 1.15427 \\ \hline
\it{Tórax Lateral \ref{fig:imgtl8}} & 1.13405 \\ \hline
\it{Tórax Lateral \ref{fig:imgtl9}} & 1.02208 \\ \hline
\it{Tórax Lateral \ref{fig:imgtl10}} & 1.27029 \\ \hline
\it{Tórax Lateral \ref{fig:imgtl11}} & 1.45903 \\ \hline
\it{Tórax Lateral \ref{fig:imgtl3}} & 2.45754 \\ \hline
\it{Tórax Lateral \ref{fig:imgtl5}} & 1.63800 \\ \hline

\end{tabular}
\label{tabla:hipervolumen-lateral}
\end{table}



% \bigskip












	% Appendix Title


 \lhead{\emph{Apéndice B}} 
 \renewcommand{\appendixname}{Apéndice}
 \chapter\bigskip
\ifpdf
  \graphicspath{{AppendixB/AppendixBFigs/PNG/}{AppendixB/AppendixBFigs/PDF/}{AppendixB/AppendixBFigs/}}
\else
  \graphicspath{{AppendixB/AppendixBFigs/EPS/}{AppendixB/AppendixBFigs/}}
\fi

\section{Frente Pareto Robusto y óptimo de cada imagen de referencia}

\begin{figure}[H]
    \centering
        \subfigure[][\label{a}Tórax Frontal \ref{fig:img1}]{\includegraphics[width=7cm]{AppendixB/AppendixBFigs/FP-torax_frontal/img1-FP-TF-robusto.png}}
        \subfigure[][\label{b}Tórax Frontal \ref{fig:img2}]{\includegraphics[width=7cm]{AppendixB/AppendixBFigs/FP-torax_frontal/img2-FP-TF-robusto.png}}
   
    \caption {Frente Pareto Entropía Local - SSIM. Tórax Frontal\\}{
    {\color{myblue} {$\bullet$}} Frente pareto robusto \\
    {\color{mygreen} {$\bullet$}} Frente pareto óptimo}
    \label{fp-robusto-optimo-tf1}
\end{figure}



\begin{figure}[H]
    \centering
        \subfigure[][\label{a} Tórax Frontal \ref{fig:img4}]{\includegraphics[width=7cm]{AppendixB/AppendixBFigs/FP-torax_frontal/img4-FP-TF-robusto.png}}
        \subfigure[][\label{b} Tórax Frontal \ref{fig:img3}]{\includegraphics[width=7cm]{AppendixB/AppendixBFigs/FP-torax_frontal/img3-FP-TF-robusto.png}}
   
   \caption {Frente Pareto Entropía Local - SSIM. Tórax Frontal\\}{
    {\color{myblue} {$\bullet$}} Frente pareto robusto \\
    {\color{mygreen} {$\bullet$}} Frente pareto óptimo}
    \label{fp-robusto-optimo-tf2}
\end{figure}


\begin{figure}[H]
    \centering
        \subfigure[][\label{a} Tórax Frontal \ref{fig:img5}]{\includegraphics[width=7cm]{AppendixB/AppendixBFigs/FP-torax_frontal/img5-FP-TF-robusto.png}}
        \subfigure[][\label{b} Tórax Frontal \ref{fig:img7}]{\includegraphics[width=7cm]{AppendixB/AppendixBFigs/FP-torax_frontal/img7-FP-TF-robusto.png}}
   
    \caption {Frente Pareto Entropía Local - SSIM. Tórax Frontal\\}{
    {\color{myblue} {$\bullet$}} Frente pareto robusto \\
    {\color{mygreen} {$\bullet$}} Frente pareto óptimo}
    \label{fp-robusto-optimo-tf3}
\end{figure}

\begin{figure}[H]
    \centering
        \subfigure[][\label{a} Tórax Frontal \ref{fig:img6}]{\includegraphics[width=7cm]{AppendixB/AppendixBFigs/FP-torax_frontal/img6-FP-TF-robusto.png}}
        \subfigure[][\label{b} Tórax Frontal \ref{fig:img8}]{\includegraphics[width=7cm]{AppendixB/AppendixBFigs/FP-torax_frontal/img8-FP-TF-robusto.png}}
   
    \caption {Frente Pareto Entropía Local - SSIM. Tórax Frontal\\}{
    {\color{myblue} {$\bullet$}} Frente pareto robusto \\
    {\color{mygreen} {$\bullet$}} Frente pareto óptimo}
    \label{fp-robusto-optimo-tf4}
\end{figure}

\begin{figure}[H]
    \centering
        \subfigure[][\label{a} Tórax Frontal \ref{fig:img9}]{\includegraphics[width=7cm]{AppendixB/AppendixBFigs/FP-torax_frontal/img9-FP-TF-robusto.png}}
        \subfigure[][\label{b} Tórax Frontal \ref{fig:img10}]{\includegraphics[width=7cm]{AppendixB/AppendixBFigs/FP-torax_frontal/img10-FP-TF-robusto.png}}
   
    \caption {Frente Pareto Entropía Local - SSIM. Tórax Frontal\\}{
    {\color{myblue} {$\bullet$}} Frente pareto robusto \\
    {\color{mygreen} {$\bullet$}} Frente pareto óptimo}
    \label{fp-robusto-optimo-tf5}
\end{figure}

\begin{figure}[H]
    \centering
        \subfigure[][\label{a} Tórax Frontal \ref{fig:img11}]{\includegraphics[width=7cm]{AppendixB/AppendixBFigs/FP-torax_frontal/img11-FP-TF-robusto.png}}
        \subfigure[][\label{b} Tórax Frontal \ref{fig:img12}]{\includegraphics[width=7cm]{AppendixB/AppendixBFigs/FP-torax_frontal/img12-FP-TF-robusto.png}}
     \caption {Frente Pareto Entropía Local - SSIM. Tórax Frontal\\}{
    {\color{myblue} {$\bullet$}} Frente pareto robusto \\
    {\color{mygreen} {$\bullet$}} Frente pareto óptimo}
    \label{fp-robusto-optimo-tf6}
\end{figure}

\begin{figure}[H]
    \centering
        \subfigure[][\label{a} Tórax Frontal \ref{fig:img13}]{\includegraphics[width=7cm]{AppendixB/AppendixBFigs/FP-torax_frontal/img13-FP-TF-robusto.png}}
        \subfigure[][\label{b} Tórax Frontal \ref{fig:img14}]{\includegraphics[width=7cm]{AppendixB/AppendixBFigs/FP-torax_frontal/img14-FP-TF-robusto.png}}
     \caption {Frente Pareto Entropía Local - SSIM. Tórax Frontal\\}{
    {\color{myblue} {$\bullet$}} Frente pareto robusto \\
    {\color{mygreen} {$\bullet$}} Frente pareto óptimo}
    \label{fp-robusto-optimo-tf7}
\end{figure}

\begin{figure}[H]
    \centering
        \subfigure[][\label{a} Tórax Frontal \ref{fig:img15}]{\includegraphics[width=7cm]{AppendixB/AppendixBFigs/FP-torax_frontal/img15-FP-TF-robusto.png}}
     \caption {Frente Pareto Entropía Local - SSIM. Tórax Frontal\\}{
    {\color{myblue} {$\bullet$}} Frente pareto robusto \\
    {\color{mygreen} {$\bullet$}} Frente pareto óptimo}
    \label{fp-robusto-optimo-tf8}
\end{figure}


\begin{figure}[H]
    \centering
        \subfigure[][\label{a} Tórax Lateral \ref{fig:imgtll}]{\includegraphics[width=7cm]{AppendixB/AppendixBFigs/FP-torax_lateral/img1-FP-TL-robusto.png}}
        \subfigure[][\label{a} Tórax Lateral \ref{fig:imgtl2}]{\includegraphics[width=7cm]{AppendixB/AppendixBFigs/FP-torax_lateral/img4-FP-TL-robusto.png}}
     \caption {Frente Pareto Entropía Local - SSIM. Tórax Lateral\\}{
    {\color{myblue} {$\bullet$}} Frente pareto robusto \\
    {\color{mygreen} {$\bullet$}} Frente pareto óptimo}
    \label{fp-robusto-optimo-tl1}
\end{figure}

\begin{figure}[H]
    \centering
        \subfigure[][\label{a} Tórax Lateral \ref{fig:imgtl4}]{\includegraphics[width=7cm]{AppendixB/AppendixBFigs/FP-torax_lateral/img4-FP-TL-robusto.png}}
        \subfigure[][\label{a} Tórax Lateral \ref{fig:imgtl6}]{\includegraphics[width=7cm]{AppendixB/AppendixBFigs/FP-torax_lateral/img6-FP-TL-robusto.png}}
    \caption {Frente Pareto Entropía Local - SSIM. Tórax Lateral\\}{
    {\color{myblue} {$\bullet$}} Frente pareto robusto \\
    {\color{mygreen} {$\bullet$}} Frente pareto óptimo}
    \label{fp-robusto-optimo-tl2}
\end{figure}

\begin{figure}[H]
    \centering
        \subfigure[][\label{a} Tórax Lateral \ref{fig:imgtl7}]{\includegraphics[width=7cm]{AppendixB/AppendixBFigs/FP-torax_lateral/img5-FP-TL-robusto.png}}
         \subfigure[][\label{a} Tórax Lateral \ref{fig:imgtl8}]{\includegraphics[width=7cm]{AppendixB/AppendixBFigs/FP-torax_lateral/img8-FP-TL-robusto.png}}
    \caption {Frente Pareto Entropía Local - SSIM. Tórax Lateral\\}{
    {\color{myblue} {$\bullet$}} Frente pareto robusto \\
    {\color{mygreen} {$\bullet$}} Frente pareto óptimo}
    \label{fp-robusto-optimo-tl3}
\end{figure}

\begin{figure}[H]
    \centering
        \subfigure[][\label{a} Tórax Lateral \ref{fig:imgtl9}]{\includegraphics[width=7cm]{AppendixB/AppendixBFigs/FP-torax_lateral/img9-FP-TL-robusto.png}}
         \subfigure[][\label{a} Tórax Lateral \ref{fig:imgtl10}]{\includegraphics[width=7cm]{AppendixB/AppendixBFigs/FP-torax_lateral/img10-FP-TL-robusto.png}}
    \caption {Frente Pareto Entropía Local - SSIM. Tórax Lateral\\}{
    {\color{myblue} {$\bullet$}} Frente pareto robusto \\
    {\color{mygreen} {$\bullet$}} Frente pareto óptimo}
    \label{fp-robusto-optimo-tl4}
\end{figure}


\begin{figure}[H]
    \centering
        \subfigure[][\label{a} Tórax Lateral \ref{fig:imgtl11}]{\includegraphics[width=7cm]{AppendixB/AppendixBFigs/FP-torax_lateral/img11-FP-TL-robusto.png}}
    \caption {Frente Pareto Entropía Local - SSIM. Tórax Lateral\\}{
    {\color{myblue} {$\bullet$}} Frente pareto robusto \\
    {\color{mygreen} {$\bullet$}} Frente pareto óptimo}
    \label{fp-robusto-optimo-tl5}
\end{figure} % Appendix Title

\backmatter
%% --------------- Referencias ------------------------------------

\renewcommand{\bibname}{Referencias}
\bibliography{References/reference_tapeyty}
\bibliographystyle{apacite}


\end{document}  % The End
%% ----------------------------------------------------------------
