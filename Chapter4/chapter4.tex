\chapter{Optimización}
\label{chap4}
\ifpdf
  \graphicspath{{Chapter4/Chapter4Figs/PNG/}{Chapter4/Chapter4Figs/PDF/}{Chapter4/Chapter4Figs/}}
\else
  \graphicspath{{Chapter4/Chapter4Figs/EPS/}{Chapter4/Chapter4Figs/}}
\fi

\markboth{\hfill \thechapter. Optimización}{\hfill \thechapter. Optimización}


\citet{Akhtar2017BacktrackingOptimization} proponen combinar un algoritmo metaheurístico con contenedores de basura inteligente equipados con diferentes sensores, el estudio introduce el concepto del umbral de nivel de residuos (TWL, \textit{Threshold Waste Level}) de los contenedores de residuos para reducir el número de contenedores que se deben vaciar, al encontrar un rango óptimo, minimizando así la distancia.

\citet{Kallel2016UsingTunisia} desarrolló escenarios optimizados utilizando la herramienta ArcGIS Network Analyst para mejorar la eficiencia de la recolección de residuos y el transporte en el distrito Cívico El Habib de la ciudad de Sfax, Túnez.

