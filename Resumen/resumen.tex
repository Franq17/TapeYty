\begin{resumen}
\textit{TapeYty} es una herramienta desarrollada para calcular rutas óptimas para vehículos de recolección de basura urbana en la ciudad de Asunción. Esta herramienta genera beneficios principalmente en los aspectos económicos y ambientales de la ciudad, donde un gran número de personas genera una alta cantidad de residuos. Esto hace que la complejidad de la gestión de basura sea aún mayor.

El problema de enrutamiento se trata como un problema de Cartero Rural Abierto que busca minimizar la distancia que deben recorrer los vehículos de recolección en su área de trabajo. Para lograr el objetivo, \textit{TapeYty} se basa en técnicas de programación matemática y Sistema de Información Geográfica (GIS) que permite que la administración de la red de rutas pueda actualizar la vía y su estado de bloqueo o no bloqueado. Esto implica que cuando los cambios de estado de las calles, \textit{TapeYty} modifica el gráfico que representa la red de carreteras y vuelve a calcular las soluciones, proporcionando nuevas rutas óptimas para cada vehículo de recolección.

La herramienta ha brindado soluciones que ahorran en promedio 20\% de distancia recorrida en comparación con el recorrido actual.
\end{resumen}