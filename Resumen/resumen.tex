\begin{resumen}
En la ciudad Asunción, la cantidad de personas que convergen a diario trae consigo una alta generación de residuos y esto hace que la complejidad en la gestión de la basura sea cada vez mayor. Se aborda el problema de enrutamiento como un problema del cartero rural abierto dirigido, minimizando la distancia a recorrer por los vehículos recolectores y luego se realiza una búsqueda en profundidad para obtener su secuencia. Para lograr el objetivo \textit{TapeYty} se basa en técnicas de programación matemáticas y herramienta GIS lo que permite la gestión de la red de rutas al poder actualizar el sentido de las calles, inhabilitar calles, agregar restricciones de giro o de continuar el sentido (contramano). Esto implica que ante cambios de estado de las calles \textit{TapeYty} re-calcula las soluciones y provee el nuevo recorrido óptimo a cada vehículo de recolección.
\end{resumen}