\begin{resumen}
\textit{TapeYty} es una herramienta desarrollada para calcular rutas óptimas para los vehículos de recolección de basura urbana en la ciudad de Asunción. Esta herramienta genera beneficios principalmente en los aspectos económicos y ambientales de la ciudad. La cantidad de personas que convergen a diario trae consigo una alta generación de residuos, esto hace que la complejidad de la gestión de basura sea aún mayor.

El problema de enrutamiento se trata como un problema del Cartero Rural Abierto Dirigido que busca minimizar la distancia que deben recorrer los vehículos de recolección en su zona de trabajo. Para lograr el objetivo, \textit{TapeYty} se basa en técnicas de programación matemática y Sistema de Información Geográfica (GIS) permitiendo la gestión de la red de rutas siendo posible actualizar el sentido de las calles y su estado de habilitado o no habilitado. Esto implica que cuando se realizan cambios de estado en las calles, \textit{TapeYty} modifica el grafo que representa la red de rutas y vuelve a calcular las soluciones, proporcionando nuevas rutas óptimas para cada vehículo de recolección.

La herramienta ha brindado soluciones que ahorran en promedio un 20\% de distancia en comparación con el recorrido actual.

% \textit{TapeYty} is a tool developed to calculate optimal paths for urban garbage collection vehicles in the Asunción city. This tool generates benefits mainly in the economic and environmental aspects of the city where a large number of people brings a high generation of waste. This makes the complexity of garbage management even greater. 

% The routing problem is treated as the open rural postman problem which seeks to minimize the distance to be traveled by the collection vehicles. To achieve the objective \textit{TapeYty} is based on mathematical programming techniques and Geographical Informatiom System (GIS) tool which allows the management of the route network, being able able to update the road way and their state of blocked or nonblocked. This implies that when changes of state of the streets \textit{TapeYty} modifies the graph that represents the road network and re-calculates the solutions providing new optimal routes to each collection vehicle. 

% The tool has provided solutions that save on average 20\% of distance traveled compared to current tour.
\end{resumen}